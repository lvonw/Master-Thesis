\chapter{Ergebnisse und Diskussion}

In diesem Kapitel werden die Ergebnisse der in Kapitel \ref{ch:Methodik} konzipierten und in Kapitel \ref{ch:Implementierung} umgesetzten Methoden und Modelle Präsentiert und systematisch auf die Erfüllung der in Abschnitt \ref{sec:Zielsetzung} definierten geprüft. \\
Dazu werden zuerst die Rekonstruktionsqualität des VAEs demonstriert, welcher Grundlage für die Generierung des LDMs ist. Daraufhin wird die Fähigkeit des LDMs überzeugende Samples zu synthetisieren anhand der definierten Generierungsprozesse kritisch geprüft. Der einzelnen Betrachtung folgt in diesem Kapitel eine zusammenfassende Betrachtung der Zielerfüllung durch die erzeugten Resultate. Hieran abgeleitet werden gesondert auch die Limitationen explizit und gesammelt aufgeführt und eingordnet. Abschliessend folgt ein Vergleich, der diese Arbeit in das Forschungsfeld der Terraingenierung mit DMs einordnet. \\
Um den Kontrast der DEM Darstellungen zu erhöhen, was der besseren Erkennbarkeit dient, werden im Folgenden, sofern nicht anders gekennzeichnet, relative Einfärbungen genutzt. Das bedeutet, dass die lokalen Minima, beziehungsweise Maxima maßgebend für Schwarz und Weiß sind, anstelle der globalen Grenzen von -1 und 1.  

%%%%%%%%%%%%%%%%%%%%%%%%%%%%%%%%%%%%%%%%%%%%%%%%%%%%%%%%%%%%%%%%%%%%%%%%%%%%%%%
% VAE
%%%%%%%%%%%%%%%%%%%%%%%%%%%%%%%%%%%%%%%%%%%%%%%%%%%%%%%%%%%%%%%%%%%%%%%%%%%%%%%
\section {Variational Autoencoder}

Der VAE hat als vortrainierte Komponente eines LDM zweierlei Aufgaben. Zum einen haben die erzeugten latenten Repräsentationen für ein DM sinnvoll zu verarbeiten zu sein. Zum anderen muss die Rekonstruktion der Daten maximal gut sein. \\
Die erste dieser Anforderungen kann nicht alleinestehend bewertet werden da dies das Training des DMs erfordert. Somit wird dieser bereich nicht weiter betrachtet und mit den Ergebnissen des LDMs als ganzes als gegeben hingenommen. Enstprechend wird im Folgenden die Rekonstruktionsqualität im Vordergrund stehen. \\
Der finale VAE wurde für 167000 Trainingsschritte bei einer Batchgröße von 64 für etwa 10 Stunden auf einer NVIDIA H100 GPU trainiert. Die folgenden Parameter der Architektur wurden hierbei nach Tests auf kleineren Testbildern mit einer Auflösung von $64\times64$ und unter dem Vorbild der Implementierung des KL-reg. ausgewählt:
\begin{table}[ht]
    \centering
    \begin{tabular}{p{0.45\textwidth} p{0.35\textwidth}}
        \hline\hline
        \thead{Parameter}           & \thead{Ausgewählter Wert}\\
        \hline
        U-Net Schichten             & 4                         \\
        Latente Dimension           & 4$\times$32$\times$32     \\
        EMA Gewicht                 & 0,995                     \\
        EMA Warmup                  & 2.000                     \\
        \hline
        Rekonstruktionsfehler       & L1 Distanz                \\
        \hline
        KL-Divergenz Gewicht        & 0,000001                  \\
        \hline
        Diskriminator Warmup        & 50.000                    \\
        Diskriminator Schichten     & 3                         \\
        Diskriminator Gewicht       & 0,5                       \\
        \hline
        LPIPS Netzwerk-Architektur  & VGG-16                    \\
        LPIPS Gewicht               & 1                         \\
        \hline\hline
    \end{tabular}
    \caption{Aufbau des finalen VAEs}
    \label{tab:vae_167k_aufbau}
\end{table} \\
Wie sich anhand der in Abbildung \ref{fig:vae_recons} präsentierten Ergebnisse erkennen lässt, ist der finale VAE in der Lage eingabedaten in Form von Terrains überzeugend zu Rekonstruieren. Große Strukturen werden durchweg akkurat und mit großer Genaugikeit widergegeben. Durch die genutzten Techniken welche die visuelle Ähnlichkeit verbessern, wie die Nutzung eines Diskriminators und LPIPS werden auch kleine Details abgebildet. Die Qualität der Rekonstruktionen ist hierbei ungeachtet der Art des Terrains oder Wertebereiches des Quellbildes so hoch, dass auf den ersten Blick keinerlei Unterschiede zu erkennen sind. 
\begin{figure}[htbp]
    \centering
    \subfloat{%
        \includegraphics[width=0.24\textwidth]{images/results/vae/source1.png}
    }
    \subfloat{%
        \includegraphics[width=0.24\textwidth]{images/results/vae/recon1.png}
    }
    \subfloat{%
        \includegraphics[width=0.24\textwidth]{images/results/vae/source2.png}
    }
    \subfloat{%
        \includegraphics[width=0.24\textwidth]{images/results/vae/recon2.png}
    }

    \vspace{-8pt}

    \subfloat{%
        \includegraphics[width=0.24\textwidth]{images/results/vae/source3.png}
    }
    \subfloat{%
        \includegraphics[width=0.24\textwidth]{images/results/vae/recon3.png}
    }
    \subfloat{%
        \includegraphics[width=0.24\textwidth]{images/results/vae/source4.png}
    }
    \subfloat{%
        \includegraphics[width=0.24\textwidth]{images/results/vae/recon4.png}
    }
    
    \caption{Verschiedene Quellbild- und Rekonstruktions-Paare. Links Quellbild, rechts Rekonstruktion}
    \label{fig:vae_recons}
\end{figure} \\
Bei einer genaueren Betrachterung der kleineren Details lassen sich jedoch vor allem bei DEMs mit hoher Varianz Komrpessionsartefakte beobachten. Besonders äußert sich dies in der bildlichen Schärfe der kleinen fraktalen Strukturen einer solchen Landschaft welche sichtbar abnimmt. Auch lassen sich auf der Detailebene dieser Terrains kleine punktuelle Artefakte erkennen, welche wahrscheinlich auf die Nutzung des Patch-GAN Diskriminators zurückzuführen sind. Abbildung \ref{fig:vae_artefacts} zeigt eine Detailaufnahme eines solchen Bereiches.
\begin{figure}[htbp]
    \centering
    \subfloat{%
        \includegraphics[width=0.24\textwidth]{images/results/vae/vae_artefact1.png}
    }
    \subfloat{%
        \includegraphics[width=0.24\textwidth]{images/results/vae/vae_artefact1a.png}
    }
    \subfloat{%
        \includegraphics[width=0.24\textwidth]{images/results/vae/vae_artefact2.png}
    }
    \subfloat{%
        \includegraphics[width=0.24\textwidth]{images/results/vae/vae_artefact2a.png}
    }
    \caption{Ausgewählte Rekonstruktion, bei welcher die Artefakte wie kleine Patches und verwaschene Details besonders stark ausgeprägt sind. Der gelb umrandete Bereich wird jeweils rechts vergrößert abgebildet. Linkes paar sind Quelldaten, rechtes paar die Rekonstruktionen}
    \label{fig:vae_artefacts}
\end{figure} \\
Trotz dieser leichten qualitativen Mängel in den Details einiger Samples lässt sich zusammenfassend feststellen, dass das implementierte VAE-GAN für eine akkurate Rekonstruktion der Terraindaten überaus gut geeignet ist. Es bietet somit eine gelungene Grundlage für die weitere Verwendung in dem LDM bei den Untersuchungen in den folgenden Abschnitten.

%%%%%%%%%%%%%%%%%%%%%%%%%%%%%%%%%%%%%%%%%%%%%%%%%%%%%%%%%%%%%%%%%%%%%%%%%%%%%%%
% Terrain Generierung
%%%%%%%%%%%%%%%%%%%%%%%%%%%%%%%%%%%%%%%%%%%%%%%%%%%%%%%%%%%%%%%%%%%%%%%%%%%%%%%
\section {Terrain Generierung}

In diesem Abschnitt werden die Ergebnisse des LDMs beleuchtet. Hierbei werden alle definierten Generierungsprozesse durchlaufen und einzelnd auf ihre Erfüllung der gestellten Anforderungen kritisch geprüft. Hieraus ergibt sich somit eine Gesamtbetrachtung des LDMs unter den definierten Gesichtspunkten. Begonnen wird allerdings mit dem Vergleich der zwei grundlegenden Optionen für die Architektur des DMs. 

%%%%%%%%%%%%%%%%%%%%%%%%%%%%%%%%%%%%%%%%%%%%%%%%%%%%%%%%%%%%%%%%%%%%%%%%%%%%%%%
\subsection {U-Net vs. DiT}

Beide Architekturen wurden auf der selben Maschine unter den gleichen Bedingungen für jeweils 10 Stunden auf einer NVIDIA H100 GPU trainiert. 
Die Architekturparameter sind unter dem Vorbild der Implementierung des DMs aus der von Rombach et al.\footnote{
    Rombach et al.: Latent Diffusion Models
    \cite{rombach2022high}
} vorgeschlagenen LDM Architektur gewählt. (Die Nutzung einer Kosinusbasierten Noise Schedule hat bei kleineren Experimenten keine ersichtlichen Verbesserungen beim Fehler oder der Visuellen Qualität erzeugt, daher wird der einfachheithalber die lineare genutzt.)
\begin{table}[ht]
    \centering
    \begin{tabular}{p{0.45\textwidth} p{0.35\textwidth}}
        \hline\hline
        \thead{Parameter}               & \thead{Ausgewählter Wert} \\
        \hline
        \# U-Net Schichten              & 4             \\
        Zeitembedding Dimension         & 1280          \\
        \# Attention Heads              & 8             \\
        Kondionsintegration             & AdaGN         \\
        Noise Schedule                  & Linear        \\
        \hline
        $L_\text{simple}$ Verlust       & MSE           \\ 
        $L_\text{vlb}$ Gewicht          & 0,001         \\
        \hline
        EMA Warmup                      & 2000          \\
        EMA Gewicht                     & 0,9999        \\
        \hline
        CFG Droupout Wahrscheinlichkeit & 0,1           \\
        CFG Gewicht                     & 3             \\
        \hline
        \# Parameter                    & 1.462.391.413 \\
        \hline\hline
    \end{tabular}
    \caption{Aufbau des DMs unter Nutzung einer U-Net Architektur}
    \label{tab:unet_aufbau}
\end{table} \\
Die Parameter des DiT sind den Parametervorgaben des leistungsfähigsten Modells von Peebles und Xie entnommen\footnote{
    Peebles, Xie: Diffusion Transformers, S. 5
    \cite{peebles2023scalable}
}:
\begin{table}[ht]
    \centering
    \begin{tabular}{p{0.45\textwidth} p{0.35\textwidth}}
        \hline\hline
        \thead{Parameter}               & \thead{Ausgewählter Wert} \\
        \hline
        \# DiT-Blöcke                   & 24            \\
        Patchgröße                      & 2             \\
        \# Attention Heads              & 16            \\
        Embedding Dimension             & 1280          \\
        Kondionsintegration             & AdaLN         \\
        Noise Schedule                  & Linear        \\
        \hline
        $L_\text{simple}$ Verlust       & MSE           \\ 
        $L_\text{vlb}$ Gewicht          & 0,001         \\
        \hline
        EMA Warmup                      & 2000          \\
        EMA Gewicht                     & 0,9999        \\
        \hline
        CFG Droupout Wahrscheinlichkeit & 0,1           \\
        CFG Gewicht                     & 3             \\
        \hline
        \# Parameter                    & 1.608.065.701 \\
        \hline\hline
    \end{tabular}
    \caption{Aufbau des DMs unter Nutzung einer DiT Architektur}
    \label{tab:dit_aufbau}
\end{table} \\
Das DM mit Dit konnte während des Training für 210 Epochen trainiert werden und erreichte einen über die letzten 200 Schritte gemittelten $L_\text{hybrid}$ von ca. 0,08. Das U-Net hingegen erreichte in der gleichen Zeit 705 Epochen bei einem über die letzten 200 Schritte gemittelten finalen $L_\text{hybrid}$ von etwa 0,05 was einer Verbesserung von etwa 40\% gegenüber des DiT Ansatzes entspricht. Hierbei ist allerdings anzumerken, dass durch die Nutzung des MSE für $L_\text{simple}$ diese Reduktion nicht einer linearen sondern quadratischen Verbesserung entspricht. Entsprechend lassen sich bei den Ergebnissen, welche in Abbildung \ref{fig:dit_unet_results} präsentiert sind, erhebliche qualitative Unterschiede feststellen. Wohingegen die einzelnen Patches des DiT ganz eindeutig zu erkennen sind, sind die Ergebnisse durchgängig kohärent und ohne solche Artefakte.
\begin{figure}[htbp]
    \centering
    \subfloat{%
        \includegraphics[width=0.32\textwidth]{images/results/dit/dit1.png}
    }
    \subfloat{%
        \includegraphics[width=0.32\textwidth]{images/results/dit/dit2.png}
    }
    \subfloat{%
        \includegraphics[width=0.32\textwidth]{images/results/dit/dit4.png}
    }
    
    \vspace{-8pt}

    \subfloat{%
        \includegraphics[width=0.32\textwidth]{images/results/unet/unet1.png}
    }
    \subfloat{%
        \includegraphics[width=0.32\textwidth]{images/results/unet/unet2.png}
    }
    \subfloat{%
        \includegraphics[width=0.32\textwidth]{images/results/unet/unet3.png}
    }
    \caption{Samples des LDM unter Nutzung eines DiTs (oben) und eines UNets (unten). Die Artefakte der Patches beim DiT sind klar erkennnbar.}
    \label{fig:dit_unet_results}
\end{figure} \\
Aufgrund dieser eindeutigen qualitativen Mängel des DiT- im direkten Vergleich zum U-Net-DM wird im weiteren Verlauf dieses Kapitels ausschließlich die U-Net Architektur verwendet und bewertet. 


%%%%%%%%%%%%%%%%%%%%%%%%%%%%%%%%%%%%%%%%%%%%%%%%%%%%%%%%%%%%%%%%%%%%%%%%%%%%%%%
\subsection {Ungesteuerte Generierung}
\label{subsec:unconditional_results}

Die Generierung ohne Skizzen soll überprüfen zu welchem Grad das LDM die Struktur der Datenverteilung unter Zuhilfenahme der zugehörigen Kontrollsignale erlernt hat. Diese wurden im Vergleich zu vorherigen Ansätzen stark vereinfacht um die Skizzenkontrolle von der Konditionierung des Modells zu lösen. Die abgebildeten Ergebnisse haben zur Vergleichbarkeit jeweils exakt die gleichen Seeds. \\
Der erste Testlauf hierzu ist die Generierung von Samples ohne jegliches Kontrollsignal. Hierbei ist nicht mit guten Ergebnissen zu Rechnen, da die Varianz ohne jegliche Eingrenzungen zu hoch ist. Trotzdem bieten die Ergebnisse, welche in Abbildung \ref{fig:unc_terrain_noclass} zu sehen sind, eine Grundlage, um die jeweiligen Veränderungen durch die Einführung der Konditionen in den folgenden Durchläufen einordnen zu können.
\begin{figure}[htbp]
    \centering
    \subfloat{%
        \includegraphics[width=0.24\textwidth]{images/results/uncontrolled/noclass1.png}
    }
    \subfloat{%
        \includegraphics[width=0.24\textwidth]{images/results/uncontrolled/noclass2.png}
    }
    \subfloat{%
        \includegraphics[width=0.24\textwidth]{images/results/uncontrolled/noclass3.png}
    }
    \subfloat{%
        \includegraphics[width=0.24\textwidth]{images/results/uncontrolled/noclass4.png}
    }
    \caption{Samples ohne jegliche Eingrenzung durch Kontrollsignale}
    \label{fig:unc_terrain_noclass}
\end{figure} \\
Durch die Nutzung des selben Seeds für die Ergebnisse sind alle von diesen Ergebnissen abweichende Bereiche allein auf die jeweiligem Kontrollsignale zurückzuführen. Die hierbei erste Betrachtung sind die unterschiedlichen Terrainklassen, ohne Abweichung bei Klima. Die Ergebnisse in Abbildung \ref{fig:unc_terrain} zeigen hierbei, dass zu einem gewissen Maß die jeweils zu erwartenden Strukturen zu Erkennen sind. Steile Berge zeigt eindeutig gewisse bergige Eigenschaften. Ebenfalls ist die Variation in den Höhenwerten selbst Sichtbar höher, als bei den anderen Ergebnissen. Allesamt zeigen ebenfalls mehr kohärente globale Struktur als ohne Kondition. Allerdings ist die Qualität der Ergebnisse trotzdem enttäuschend gering. Globale Strukturen lassen sich mit Ausnahme der steilen Berge kaum erkennen. Auch sind die strukturellen unterschiede zwischen Hügel, Plateau und Flachland nur sehr gering. 
\begin{figure}[htbp]
    \centering
    \subfloat{%
        \includegraphics[width=0.24\textwidth]{images/results/uncontrolled/unc_terrain1.png}
    }
    \subfloat{%
        \includegraphics[width=0.24\textwidth]{images/results/uncontrolled/unc_terrain5.png}
    }
    \subfloat{%
        \includegraphics[width=0.24\textwidth]{images/results/uncontrolled/unc_terrain10.png}
    }
    \subfloat{%
        \includegraphics[width=0.24\textwidth]{images/results/uncontrolled/unc_terrain15.png}
    }
    \caption{Samples von moderatem Klima mit variierender Terrainklasse. Um die unterschiede der Terrainhöhen hervorzuheben handelt es sich hier um eine absolute Einfärbung. Von links nach Rechts: Steile Berge, Hügel, Plateau, Flachland.}
    \label{fig:unc_terrain}
\end{figure} \\
Ein ähnliches Bild zeichnet sich bei der gesonderten Betrachtung der Klima-Einflüsse in Abbildung \ref{fig:unc_climate}. Auch hier sind erlernte Eigenschaften durch die Unterscheidung zu den vorherigen Resultaten klar erkennbar. Trotzdem sind die Ergebnisse dominiert durch kleine Strukturen, mit nur sehr wenigen erkennbaren globalen Zusammenhängen.
\begin{figure}[htbp]
    \centering
    \subfloat{%
        \includegraphics[width=0.24\textwidth]{images/results/uncontrolled/unc_climate1.png}
    }
    \subfloat{%
        \includegraphics[width=0.24\textwidth]{images/results/uncontrolled/unc_climate5.png}
    }
    \subfloat{%
        \includegraphics[width=0.24\textwidth]{images/results/uncontrolled/unc_climate12.png}
    }
    \subfloat{%
        \includegraphics[width=0.24\textwidth]{images/results/uncontrolled/unc_climate28.png}
    }
    \caption{Samples von Steilen Bergen mit variierender Klimaklasse. Von links nach Rechts: Tropisch, Trocken, Moderat, Kalt.}
    \label{fig:unc_climate}
\end{figure} \\
Die vermutete Begründung für diese Mängel ist eine fehlende Aussagekraft der Kontrollsignale, beziehungsweise eine zu hohe Varianz der möglichen Ergebnisse. Dies erklärt weshalb zwar erkennbare Unterschiede durch die Konditionen resultieren, diese aber trotzdem keine überzeugenden Ergebnisse zur Folge haben. Die Erkenntniss die hieraus unvermeidlich ist, ist dass das grundlegende Konzept der gewählten Klassen in zukünftigen Iterationen überdacht werden muss.  
\begin{figure}[htbp]
    \centering
    \subfloat{%
        \includegraphics[width=0.40\textwidth]{images/results/uncontrolled/unc_climate1.png}
    }
    \subfloat{%
        \raisebox{0.25\height}{\includegraphics[width=0.40\textwidth]{images/results/uncontrolled/unc_climate1r.png}}
    }
    % \subfloat{%
    %     \includegraphics[width=0.24\textwidth]{images/results/uncontrolled/unc_climate28.png}
    % }
    % \subfloat{%
    %     \raisebox{0.25\height}{\includegraphics[width=0.24\textwidth]{images/results/uncontrolled/unc_climate28r.png}}
    % }
    \caption{Sample und korrespondierendes Rendereing im Detail}
    \label{fig:unc_climate_renders}
\end{figure}


%%%%%%%%%%%%%%%%%%%%%%%%%%%%%%%%%%%%%%%%%%%%%%%%%%%%%%%%%%%%%%%%%%%%%%%%%%%%%%%
\subsection {Skizzenbasierte Generierung}

Nach den mangelhaften Ergebnissen der ungesteuerten Generation ist nun zu überprüfen in wiefern das Modell durch Skizzen kontrollierbar ist. In vorherigen Ansätzen wurde die definition der globale Struktur der Landschaften durch die Kontrollsignale vorgenommen, während sie hier gesondert angegangen wird. Entsprechend ist zu erwarten, dass die Ergebnisse hier eine deutlich höhere Qualität aufweisen als zuvor, sofern der gewählte Ansatz zur Kontrolle grundsätzlich geeignet ist. \\
Abbildung \ref{fig:sketch_strengths} zweigt die genutzte Skizze sowie ihre Nutzung bei der Generierung zu drei unterschiedlichen Startzeitschritten. Hierbei ist zunächst eindeutig eine enorme Verbesserung der Samplequalität festzustellen. Gebirgshänge und andere große Strukturen sind klar zu erkennen und fügen sich glaubwürdig in den Rest des DEMs ein. Ebenfalls ist einfach Sichtbar, wie sich die Kontrolle der Skizze in den jeweiligen Zeitschritten auswirkt. Von einer sehr starken übereinstimmung, bei welcher trotzdem Landschaftseigenschaften zu erkennen sind, bishin zu einer nur zu erahnenden Orientierung.  
\begin{figure}[htbp]
    \centering
    \subfloat{%
        \includegraphics[width=0.24\textwidth]{images/results/sketch/sketch.png}
    }
    \subfloat{%
        \includegraphics[width=0.24\textwidth]{images/results/sketch/str900.png}
    }
    \subfloat{%
        \includegraphics[width=0.24\textwidth]{images/results/sketch/str800.png}
    }
    \subfloat{%
        \includegraphics[width=0.24\textwidth]{images/results/sketch/str700.png}
    }
    \caption{Skizzenbasierte Samples mit drei unterschiedlichen Startzeitschritten. Von links nach rechts: Skizze, 900, 800, 700}
    \label{fig:sketch_strengths}
\end{figure} \\
Folgend wurden auch hier die Betrachtung der jeweiligen Auswirkungen der Kontrollsignale überprüft. Beginnend mit den Terrainklassen, siehe Abbildung \ref{fig:sketch_terrains}. Ausgangen wird hierbei von Startzeitschritt $t=800$. Die Ergebnisse sind hierbei wenig aussagekräftig im Bezug auf die Klasse. Die Skizze hat einen deutlich höheren Einfluss auf die jeweiligen Höhenwerte als die Terrainkategorie. Es lässt sich trotzdem im Ansatz eine Auswirkung in der Diffusität der extremen Eigenschaften erkennen. Während bei den steilen Bergen der Gebirgskamm noch klar definiert ist, breitet sich dieser zunehmend aus und verfranst.  
\begin{figure}[htbp]
    \centering
    \subfloat{%
        \includegraphics[width=0.24\textwidth]{images/results/sketch/sk_terrain_800_1.png}
    }
    \subfloat{%
        \includegraphics[width=0.24\textwidth]{images/results/sketch/sk_terrain_800_5.png}
    }
    \subfloat{%
        \includegraphics[width=0.24\textwidth]{images/results/sketch/sk_terrain_800_10.png}
    }
    \subfloat{%
        \includegraphics[width=0.24\textwidth]{images/results/sketch/sk_terrain_800_15.png}
    }
    \caption{Skizzenbasierte Samples mit variierender Terrainklasse. Von links nach Rechts: Steile Berge, Hügel, Plateau, Flachland.}
    \label{fig:sketch_terrains}
\end{figure} \\
Bei den Klimaklassen ist die Einflussnahme schon deutlich klarer, vermutlich, da diese die Einflüsse durch geomorphologische Prozesse der Erosion abbilden und somit mehr die Details bestimmen als die großen Strukturen. Je nach definierter Kondition sind hier die Ergebnisse stark variierend und eindeutig von einander zu unterscheiden, wie anhhand der Samples in Abbildung \ref{fig:sketch_climates} gut zu erkennen ist. 
\begin{figure}[htbp]
    \centering
    \subfloat{%
        \includegraphics[width=0.24\textwidth]{images/results/sketch/sk_climate_800_1.png}
    }
    \subfloat{%
        \includegraphics[width=0.24\textwidth]{images/results/sketch/sk_climate_800_5.png}
    }
    \subfloat{%
        \includegraphics[width=0.24\textwidth]{images/results/sketch/sk_climate_800_12.png}
    }
    \subfloat{%
        \includegraphics[width=0.24\textwidth]{images/results/sketch/sk_climate_800_28.png}
    }
    \caption{Skizzenbasierte Samples mit variierender Klimaklasse. Von links nach Rechts: Tropisch, Trocken, Moderat, Kalt.}
    \label{fig:sketch_climates}
\end{figure} \\
Durch die DEM basierte Skizzenkontrolle im Vergleich zu Signaturen ist es möglich die Generierung auf Skizzen wie Rauschen zu basieren. Dies erzeugt bei bereits groben Orientierungen wie bei einem Startzeitschritt von $t=750$ überzeugende Landschaften, siehe Abbildung \ref{fig:sketch_perlin}. Dies öffnet die Möglichkeit die schwach erlernten globalen Strukturen durch eine solche grobe Strukturgebung zu approximieren. Hierdurch ist an dieser Stelle keine explizite Erstellung einer Skizze erforderlich und trotzdem werden großflächige Eigenschaften glaubwürdig generiert. Somit können die in Unterabschnitt \ref{subsec:unconditional_results} berichteten Problem etwas konterkariert werden. Natürlich sei an dieser Stelle trotzdem erwähnt, dass aussagekräftigere Kontrollsignale die Qualität und Steuerbarkeit nur erhöhen würden.
\begin{figure}[htbp]
    \centering
    \subfloat{%
        \includegraphics[width=0.32\textwidth]{images/results/sketch/sk_perlin_750s.png}
    }
    \subfloat{%
        \includegraphics[width=0.32\textwidth]{images/results/sketch/sk_perlin_750.png}
    }
    \subfloat{%
        \includegraphics[width=0.32\textwidth]{images/results/sketch/sk_perlin_750r.png}
    }
    \caption{Skizzenbasiertes Sample mit Perlin Rauschen als Skizze und Startzeitschritt 750.}
    \label{fig:sketch_perlin}
\end{figure}

%%%%%%%%%%%%%%%%%%%%%%%%%%%%%%%%%%%%%%%%%%%%%%%%%%%%%%%%%%%%%%%%%%%%%%%%%%%%%%%
\subsection {Unendliche Generierung}

Der finale zu untersuchende Bereich der Generierung ist die unendliche Genierung. Um in diesen Ergebnissen großflächige Strukturen zu erzeugen werden auf basis der in den vorherigen Unterabschnitt gewonnenen Erkenntissen Perlin-Rauschbilder als Skizzenverwendet. Zunächst werden diese kleineren Rauschbilder aus benachbarten Zellen entnommen, womit sie ein kohärentes ganzheitliches Bild ergeben. Im weiteren Verlauf wird allerdings diese Flüssigkeit aufgebrochen um die Fähigkeit dieser Methode mit solchen Fällen umzugehen geprüft wird. \\
Zunächst allerdings folgt die Untersuchung der unterschiedlichen definierten Alphafunktionen der Masken. Wie sich an den in Abbildung \ref{fig:masks} dargelegten Ergebnissen erkennen lässt weist lediglich die Sinusfunktion keine sichtbaren Artefakte auf. Bei allen anderen sind die Übergänge einzelner Zellen klar zu erkennen. 
\begin{figure}[htbp]
    \centering
    \subfloat{%
        \includegraphics[width=0.24\textwidth]{images/results/infinite/2x2_none.png}
    }
    \subfloat{%
        \includegraphics[width=0.24\textwidth]{images/results/infinite/2x2_lin.png}
    }
    \subfloat{%
        \includegraphics[width=0.24\textwidth]{images/results/infinite/2x2_exp.png}
    }
    \subfloat{%
        \includegraphics[width=0.24\textwidth]{images/results/infinite/2x2_cos.png}
    }
    \caption{Unendliche generierung unter Nutzung unterschiedlicher Alphafunktionen für die Masken in einem $2\times2$ . Von links nach rechts: Konstant, linear, exponential, Sinus}
    \label{fig:masks}
\end{figure} \\
Tatsächlich sind jegliche Übergänge nur im direkten Vergleich zu den anderen Ansätzen in der groben Struktur von Gebirgskämmen zu erahnen. Vergleicht man das generierte Terrain jedoch nur mit der zugrundeliegenden Skizze sind diese nicht zu Erkennen, siehe Abbildung \ref{fig:cos_detail}. Dieser Ansatz ist somit eindrucksvoll in der Lage schlüssige und überzeugende Übergänge zwischen bereits Generiertem und aktuellen Samples zu erschließen.
\begin{figure}[htbp]
    \centering
    \subfloat{%
        \includegraphics[width=0.40\textwidth]{images/results/infinite/2x2_sk.png}
    }
    \subfloat{%
        \includegraphics[width=0.40\textwidth]{images/results/infinite/2x2_cos.png}
    }
    \caption{Unendliche generierung unter Nutzung der Sinusfunktion im Detail}
    \label{fig:cos_detail}
\end{figure} \\
Eine Beobachtung die bei genauerer Betrachtung jedoch auffällt, ist das die erste generierte Zelle sich in ihrer Struktur deutlich von den restlichen unterscheidet. Dies ist klar in der oberen linken Ecken des ersten Samples in der Abbildung \ref{fig:1st_regen} zu erkennen. Diese Zelle weist deutlich tiefere und zerfranstere Merkmale in der Bergstruktur im vergleich zum Rest auf, obwohl die Kontrollsignale gleich geblieben sind. Die Vermutung für die Begründung hiervon ist, dass die bereits vorgegebenen Ränder der Zelle die Freiheitsgrade und somit die Varianz der Samples deutlich einschränken. Da die erste Zelle diese Bedingung nicht hat, könnte dies zu den beobachteten Unterschieden führen. \\ 
Um dies zu überprüfen wird die erste Zelle nach abgeschlossener Generierung des gesammten Gitters einmalig neu gesampelt. Somit würde sie bei diesem durchlauf ebenfalls in der Varianz eingegrenzt. Wie sich am zweiten Sample in Abbildung \ref{fig:1st_regen} erkennen lässt, hat dies den fraglichen Bereich stark verändert, sodass er sich nun deutlich mehr in den Rest der Landschaft eingliedert. 
\begin{figure}[htbp]
    \centering
    \subfloat{%
        \includegraphics[width=0.40\textwidth]{images/results/infinite/3x3_norm.png}
    }
    \subfloat{%
        \includegraphics[width=0.40\textwidth]{images/results/infinite/3x3_regen.png}
    }
    \caption{Unendliche Generierung im $3\times3$ Gitter mit Sonderbehandlung der ersten Zelle. Links ohne Sonderbehandlung, rechts mit Sonderbehandlung}
    \label{fig:1st_regen}
\end{figure} \\
Bisher hat die unendliche Generierung unter einem kohärenten Perlin-Rauschbild sehr gute Ergebnisse erzeugt. Nun stellt sich allerdings die Frage in wiefern die Kohärenz der Skizze zu hierzu beträgt. Schließlich erscheint es möglich, dass nur durch die kontinierliche Struktur der Skizze glaubwürdige Übergänge gefunden werden können. \\
Damit dies untersucht werden kann werden nun weitentfernte Zellen des Rauschens als Skizzen für die einzelnen Landschaftszellen genommen. Um die Kohärenz zusätzlich zu unterbrechen werden sie außerdem mit einem zufälligen Gewicht skalliert um auch die Wertebereiche unterschiedlich zu halten. Ein, in Abbildung \ref{fig:disconnected} dargestelltes, Ergebnis dieses Experiments zeigt, dass auch in diesem Fall dieselbe Samplequalität erreicht wird. Dies untermauert die Leistungsfähigkeit des LDMs als auch der vorgestellten Methode zur unendlichen Generierung.  
\begin{figure}[htbp]
    \centering
    \subfloat{%
        \includegraphics[width=0.40\textwidth]{images/results/infinite/disconnected_sk.png}
    }
    \subfloat{%
        \includegraphics[width=0.40\textwidth]{images/results/infinite/disconnected_sampl.png}
    }

    \vspace{-8pt}

    \subfloat{%
        \includegraphics[width=0.70\textwidth]{images/results/infinite/disconnected_r.png}
    }
    \caption{Unendliche generierung mit nicht flüssigen Zellübergängen in der Skizze}
    \label{fig:disconnected}
\end{figure}


%%%%%%%%%%%%%%%%%%%%%%%%%%%%%%%%%%%%%%%%%%%%%%%%%%%%%%%%%%%%%%%%%%%%%%%%%%%%%%%
% Evaluation
%%%%%%%%%%%%%%%%%%%%%%%%%%%%%%%%%%%%%%%%%%%%%%%%%%%%%%%%%%%%%%%%%%%%%%%%%%%%%%%
\section{Evaluation}

In diesem Abschnitt werden die aus den Ergebnissen gewonennen Erkenntnisse noch einmal Zusammengefasst betrachtet. Sie sollen somit in Zusammenhang gesetzt werden.  

%%%%%%%%%%%%%%%%%%%%%%%%%%%%%%%%%%%%%%%%%%%%%%%%%%%%%%%%%%%%%%%%%%%%%%%%%%%%%%%
\subsection{Limitation}

Aus der Betrachtung der erzielten Ergebnisse wurden einige Kernlimitationen des vorgeschlagenen LDMs offensichtlich. Diese werden an dieser Stelle noch einmal klar in Äußerung und möglicher Ursache zusammengefasst.

%%%%%%%%%%%%%%%%%%%%%%%%%%%%%%%%%%%%%%%%%%%%%%%%%%%%%%%%%%%%%%%%%%%%%%%%%%%%%%%
\subsubsection{Geringe Aussagekraft der Kontrollsignale}

Wie bereits geschildert ist die Qualität der generierten Terrainsamples ohne jegliche Steuerung durch Skizzen enttäuschend gering. Die vermutete Begründung ist hierfür, dass die, für die Ermöglichung der Skizzensteuerung durch DEMs anstelle von Signaturen, simplifizierte Konditionierung des Modells verglichen mit bisherigen Ansätzen, nicht ausreichend aussagekräftig ist, um kohärente Ergebnisse zu Erzeugen. 

%%%%%%%%%%%%%%%%%%%%%%%%%%%%%%%%%%%%%%%%%%%%%%%%%%%%%%%%%%%%%%%%%%%%%%%%%%%%%%%
\subsubsection{Geringer Detailgrad}

Die Konzeption des LDMs sah vor großflächige Landschaften zu generieren. Mit einer entsprechenden Sampleauflösung von $256\times256$ und folglich einer seitenlänge eines einzelnen Höhenwertes von ca. 220 Metern ist hier kein hoher Detailgrad zu erwarten. Nichtsdestotrotz hat diese grobe Auflösung zur Folge, dass die Landschaften in ihrer generierten Form nicht ohne Weiteres für eine Verwendung in der Praxis genutzt werden können. \\
Da dies keineswegs ein für das Anwendungsgebiet der Terraingenerierung exklusives Problem ist sondern für DMs im generellen gilt, existieren hierfür bereits mehrere Lösungsansätze. Die vielleicht am besten Untersuchte Methoden dieses Problem zu Lösen sind Upscaling-Ansätze. Diese werden auch bereits in anderen Ansätzen zur Terraingeneriung vorgeschlagen wie von Lochner et al.\footnote{
    Vgl. Lochner et al.: Interactive Terrain Authoring using Diffusion Models, S. 3
    \cite{lochner2023interactive}
} \\
Eine Integration eines solchen Upscaling Modells wäre also für eine weitere Verfolgung des hier vorgestellten LDMs eine geeignete Lösung. Dies könnte ebenfalls eine Nutzung in der Praxis ermöglichen.   

%%%%%%%%%%%%%%%%%%%%%%%%%%%%%%%%%%%%%%%%%%%%%%%%%%%%%%%%%%%%%%%%%%%%%%%%%%%%%%%
\subsection{Zielerfüllung}

Im folgenden werden die bisher gewonnenen Erkenntnisse konkret für die Bewertung der Erfüllung der Ziele, die diese Arbeit umrahmen, Zusammengefasst. Dabei wird jedes einzelne Kernziel betrachtet und das Maß der jeweiligen Erfüllung anhand der vorgestellten Ergebnisse begründet.

\begin{enumerate}
    \item \textbf {Entwicklung eines leistungsfähigen Variatonal-Autoencoders:} \\
    Der entwickelte VAE nutzt die in der Praxis etablierten Techniken zur Verbesserung der Rekonstruktion und der Eignung für LDMs. Entsprechend weisen die Ergebnisse eine sehr hohe Rekonstruktionsqualität, vor allem bei großen Strukturen auf. Bei kleinen Details erweisen sich zwar die für VAEs üblichen und zu erwartenden Schwächen auf auch lassen sich hier Artefakte des Patch-GAN Ansatzes erkennen. Insgesamt ist die Bewertung der Ergebnisse als sehr gut gelungen einzustufen, weswegen dieses Ziel als vollumfänglich erfüllt angesehen wird. 
    
    \item \textbf {Erstellung eines mächtigen Latent Diffusion Modells:} \\
    Es wurde ein LDM vorgestellt das auf Methoden welche dem aktuellen Stand der Technik entsprechen basiert. Darunter fallen unter Anderem die Nutzung eines VAE-GAN, das Erlernen der Varianz und die Nutzung von Classifier Free Guidance. Weitere übliche Techniken wie eine Kosinus Noise-Schedule oder DiT wurden ausprobiert und für die weitere Verwendung als ungeeignet oder unnütz befunden. \\
    Das LDM kann vorallem auf Basis von Skizzen sehr gute Samples synthetisieren. Es enttäuscht allerdings bei der Generierung ohne eine solche Kontrolle mit qualitativ unbrauchbaren Ergebnissen. Dies liegt an der ungenügenden Fähigkeit des Modells große Strukuren eigenständig wiederzugeben. Die zugrundeliegende Skizze kann allerdings sehr simpel sein und nur schwach umgesetzt werden weswegen dieses Ziel im Ganzen als erfüllt angesehen wird, wenngleich mit der Einschränkung bei unktrollierter Generierung.

    \item \textbf {Intuitive Steuerung der Generierung durch Skizzen:} \\
    Die Nutzung von DEMs als Skizzen hat sich als sehr einfache, intuitive und mächtige Art der Steuerung bewiesen. Durch sie ist es, auf einer zuvor unmöglichen Weise, möglich detaillierte sowie grobe Skizzen in eine, jeweils angemessenen Maß zu verarbeiten. Durch diesen Ansatz ist es ebenfalls erstmals möglich Rauschbilder wie Perlin-Rauschen als Skizzen zu verwenden, da keinerlei Landschaftssignaturen sondern lediglich Höhenwerte verwendet werden. \\
    Aufgrund der sehr guten Ergebnisse in allen untersuchten Bereichen wird dieses Ziel als vollständig erfüllt bewertet. 

    \item \textbf {Nahtlose Zellbasierte unendliche Generierung:} \\
    Durch den vorgestellten innovativen Ansatz zur unendliche Generierung welcher auf der Nutzung der Inpainting-Technik von DMs beruht ist erzeugt bei geeigneter Definition der Maske wie der vorgeschlagene Sinus-Ansatz durchweg überzeugende Ergebnisse, welche selbst bei komplett unterschiedlichen benachbarten Zellen glaubwürdige und kohärente Übergänge erzeugen kann. Nähte lassen in extremen Fällen zwar in großen strukturen erahnen, bei Betrachtung auf kleiner Ebene sind diese allerdings nicht mehr zu erkennen. \\
    Aufgrund dieser Ergebnisse ist dieses Ziel als vollumfänglich erfüllt anzusehen.     

\end{enumerate}
Zusammenfassend wurden alle gesetzten Kernziele erfüllt. Die ungesteuerte Genierung bleibt zwar hinter allen Erwartungen zurück und ist somit bei der Bewertung der Leistungsfähigkeit des vorgestellten LDMs ein wesentlicher Kritikpunkt in allen anderen Bereichen wurden jedoch qualitativ sehr überzeugende Ergebnisse erzielt. 
 
%%%%%%%%%%%%%%%%%%%%%%%%%%%%%%%%%%%%%%%%%%%%%%%%%%%%%%%%%%%%%%%%%%%%%%%%%%%%%%%
\subsection{Vergleich}

Ein qualitativer Vergleich zu bisherigen Ansätzen unter der Nutzung von DMs fällt aus unterschiedlichen Gründen schwer. Zum einen ist die Zielsetzung dieser Arbeit was den Anspruch an die Generierung angeht grundsätzlich eine Andere. Dies hat als Auswirkung, dass die Genutzten Daten nicht speziell ausgewählt wurden, um faire Vergleiche zwischen vorgeschlagenen Methoden zu ermöglichen, sondern um die Lösung der gestellten Probleme möglichst gut umsetzten zu können. \\
Zum anderen sind Quellcode und Gewichte der Diffusionsmodelle bei keiner der Veröffentlichungen zugänglich gemacht worden. Somit ist keine eigene Einschätzung oder Anpassung der Testbedingung möglich. Die einzigen Möglichkeit eines Vergleiches bestehen somit aus den Referenzangaben welche aus den Publikationen selbst entnommen werden können, und vergleiche der Methodiken und ihrer beschriebenen Vor- beziehungsweise Nachteile. 




