\chapter{Grundlagen}
\label{ch:Grundlagen}

In diesem Kapitel werden die Grundlagen der verwendeten Themen und Technologien --- Begriffe und Sachverhalte --- erläutert, die für das Verständnis der folgenden Arbeit notwendig sind.
Die Grundlagen werden ohne Bezug auf das konkrete Projekt, allgemein dargestellt. Sie sind zielgerichtet auf das Verständnis des Hauptteils bezogen.



% \paragraphnl ist ein custom command. Ein \paragraph plus ein line-break / new-line.
% (Nach \paragraphnl{Paragraph} muss eine Zeile frei gelassen werden.)


%%%%%%%%%%%%%%%%%%%%%%%%%%%%%%%%%%%%%%%%%%%%%%%%%%%%%%%%%%%%%%%%%%%%%%%%%%%%%%%
% Mathematics
%%%%%%%%%%%%%%%%%%%%%%%%%%%%%%%%%%%%%%%%%%%%%%%%%%%%%%%%%%%%%%%%%%%%%%%%%%%%%%%
\section{Mathematik}

Der Abschnitt zur Mathematik dient sowohl zur Einführung neuer Konzepte sowie zur Definition der in dieser Arbeit verwendeten Nomenklatur von eventuell bereits Bekanntem.

%%%%%%%%%%%%%%%%%%%%%%%%%%%%%%%%%%%%%%%%%%%%%%%%%%%%%%%%%%%%%%%%%%%%%%%%%%%%%%%
\subsection{Wahrscheinlichkeitsdichtefunktionen}
Eine Wahrscheinlichkeitsdichtefuntion (WDF) ist eine reelle Funktion $p$, welche die Wahrscheinlichkeit beschreibt, dass eine Zufallsvariable X in einem bestimmten Bereich liegt. Diese Wahrscheinlichkeit wird wie folgt berechnet:
\begin{equation}
    a,b \in X \land a \le b, \space p([a, b]) = \int_a^b p(x) dx
\end{equation}
Die WDF hat dabei folgende zwei Invarianten zu erfüllen\footnote{
    Deisenroth: Mathematics for Machine Learning, S. 181
    \cite{Deisenroth2020}
}:
\begin{enumerate}
    \item Sie ist über den gesamten Definitionsbereich nicht negativ. 
    \begin{equation}
        \forall x \in X, \space p(x) \ge 0
    \end{equation}
    \item Das Integral der WDF-Funtion ist berechenbar und über den Definitionsbereich gleich eins.
    \begin{equation}
        \int p(x) dx = 1
    \end{equation}
\end{enumerate}

Oft kann es sein, dass die Parametrisierung angegeben werden soll. Dies wird im weiteren Verlauf dieser Arbeit in der Regel als $p_\theta(X)$ denotiert, wobei in diesem Fall $\theta$ die Parametrisierung ist. Der Vollständigkeithalber sei erwähnt, dass einige Publikationen eine alternative Schreibweise $p(X; \theta)$ wählen. \\
Desweiteren wird im Bereich des Machine Learning oft die Entnahme einer Stichprobe $x$ aus einer WDF $p$, was auch Sampling gennant wird, $x \sim p$ gekennzeichnet. 


%%%%%%%%%%%%%%%%%%%%%%%%%%%%%%%%%%%%%%%%%%%%%%%%%%%%%%%%%%%%%%%%%%%%%%%%%%%%%%%
\subsection{Multidimensionale gaußsche Normalverteilung}

Die gaußsche Normalverteilung $\mathcal N$, oder einfach Normalverteilung ist eine der bekanntesten kontinuierlichen Wahrscheinlichkeitsverteilungen. Definiert ist sie im mehrdimensionalen Fall mit $d$ Dimensionen durch ihre WDF:
\begin{equation}
    \mathcal N(x; \mu, \Sigma) = 
    \frac{1}{\sqrt{(2\pi)^d \det{\Sigma}}}
    e^{-\frac{1}{2}(x-\mu)^T \Sigma^{-1} (x-\mu)}
\end{equation}
$\mu$ ist hierbei der Erwahrtungswert und $\Sigma$ die Kovarianzmatrix. \\
Eine besondere Variante ist hierbei die Standardnormalverteilung $\mathcal N(x; 0, I)$ welche oft einfach durch $\mathcal N$ abgekürzt wird.

%%%%%%%%%%%%%%%%%%%%%%%%%%%%%%%%%%%%%%%%%%%%%%%%%%%%%%%%%%%%%%%%%%%%%%%%%%%%%%%
\subsection{Erwartungswert}

Der Erwartungswert einer Zufallsvariable $X$ unter einer WDF $p(X)$ ist gegeben als: 
\begin{equation}
    \mathbb E(X) := \int x p(x) dx
\end{equation}
In vielen Publikationen im Bereich der künstlichen Intelligenz wird der Erwartungswert auch wie folgt angegeben:
\begin{equation}
    \mathbb E_{x \sim p}[x] := \int x p(x) dx
\end{equation}
Diese Definition kann ebenfalls um den Erwartungswert einer Funktion $f(x)$ angewandt werden, wenn x aus der Wahrscheinlichkeitsverteilung $p(X)$ entnommen wird. 
\begin{equation}
    \mathbb E_{x \sim p}[f(x)] := \int f(x) p(x) dx
\end{equation}

%%%%%%%%%%%%%%%%%%%%%%%%%%%%%%%%%%%%%%%%%%%%%%%%%%%%%%%%%%%%%%%%%%%%%%%%%%%%%%%
\subsection{Likelihood}

Die Likelihood $p(y|x)$ ist in der Bayesischen Statistik, vereinfacht ausgedrückt, eine Aussage über die Wahrscheinlichkeit, dass die Beobachtung $x$ der WDF $p$ entnommen wurde\footnote{
    Deisenroth: Mathematics for Machine Learning, S. 185
    \cite{Deisenroth2020}
}. \\
Im bereich des Machine Learning ist es oft das Ziel, die Likelihood einer parametrisierten WDF zu maximieren\footnote{
    Vgl. Goodfellow, Bengio, Courville: Deep Learning, S. 133
    \cite{Goodfellow-et-al-2016}
}. 
\begin{equation}
    \arg\max_\theta p_\theta(y|x)
\end{equation}
Dies entspricht sinngemäß der Parametrisierung der WDF, unter welcher, die beobachteten Daten, beziehungsweise die Trainingsdaten maximal wahrscheinlich sind. \\
Diese Maximierung entspricht $\arg\max_\theta \log p_\theta(y|x)$ welche auch als die Maximierung der \textit{Log-Likelihood} bezeichnet wird. Diese Formulierung wird der einfachen Likelihood in der Praxis oft vorgezogen, da sie bei der Differenzierung Vorteile hat.

%%%%%%%%%%%%%%%%%%%%%%%%%%%%%%%%%%%%%%%%%%%%%%%%%%%%%%%%%%%%%%%%%%%%%%%%%%%%%%%
\subsection{Score}

Der \textit{Score} $s$ ist der Gradient, also der Vektor aller partiellen Ableitungen der Log-Likelihood einer WDF $p$\footnote{
    Hyvärinen: Estimation of Non-Normalized Statistical Models by Score Matching, S. 2
    \cite{JMLR:v6:hyvarinen05a}
}: 
\begin{equation}
    s_\theta(x) = \frac{\partial \log p_\theta(x)}{\partial \theta} 
    = \nabla_x p_\theta(x)
\end{equation}
Der Score ist somit zu einem lokalen Maximum der WDF gerichtet, gibt also an, wie sich ein Eingabedatum verändern muss, um unter der zugrundeliegenden Wahrscheinlichkeitsverteilung maximal wahrscheinlich zu sein.

%%%%%%%%%%%%%%%%%%%%%%%%%%%%%%%%%%%%%%%%%%%%%%%%%%%%%%%%%%%%%%%%%%%%%%%%%%%%%%%
\subsection{Kullback-Leibler-Divergenz}

Die Kullback-Leibler-Divergenz, kurz KL-Divergenz, ist eine Metrik welche die Ähnlichkeit zwischen zwei Wahrscheinlichkeitsverteilungen misst. Seien $p(x)$ und $q(x)$ zwei WDFs, so ist die KL-Divergenz definiert als\footnote{
    Vgl. Goodfellow, Bengio, Courville: Deep Learning, S. 74 f.
    \cite{Goodfellow-et-al-2016}
}:
\begin{equation}
    D_\text{KL}(p||q) 
    := \mathbb{E}_{x \sim p} \left [
    \log \frac{p(x)} {q(x)}
    \right ]
    = \mathbb{E}_{x \sim p} \left [ 
        \log p(x) - \log q(x)
        \right ]
\end{equation}
Sie gibt an, wie viele extra Nats\footnote{
    Nats sind Informationseinheiten in einem Zahlsystem mit Basis $e$. Vergleichbar mit Bits in einem Binärsystem. 
} nötig sind, wenn eine für $q$ optimierte Verschlüsselung, $p$ kodiert. \\
Sie bietet einige nützliche Eigenschaften, insbesondere, dass sie nie negativ und nur dann gleich null ist, wenn $p = q$ gilt. Dies und die Tatsache, dass die Minimierung der KL-Divergenz äquivalent zur Maximierung der Log-Likelihood ist\footnote{
    Amari: Information Geometry and Its Applications, S. 48
    \cite{10.5555/3019383}
}, machen sie zu einer oft angewandten Metrik im Machine Learning. \\
Der Vollständigkeithalber sei erwähnt, dass die KL-Divergenz allerdings per Definition kein Distanzmaß ist, da sie im Allgemeinen keine kommutative Operation ist.  
\begin{equation}
    D_\text{KL}(p||q) \ne D_\text{KL}(q||p) 
\end{equation}


%%%%%%%%%%%%%%%%%%%%%%%%%%%%%%%%%%%%%%%%%%%%%%%%%%%%%%%%%%%%%%%%%%%%%%%%%%%%%%%
% AI
%%%%%%%%%%%%%%%%%%%%%%%%%%%%%%%%%%%%%%%%%%%%%%%%%%%%%%%%%%%%%%%%%%%%%%%%%%%%%%%

\section{Generative Künstliche Intelligenz}

Modelle der generativen KI versuchen im Allgemeinen die Wahrscheinlichkeitsverteilung der Eingabedaten, oft auch als Datenverteilung bezeichnet, zu modellieren. Sie unterscheiden sich somit von diskreminierenden Ansätzen, welche lediglich Eigenschaften dieser Verteilung erlernen. \\
Es kann schwierig fallen, intuitiv die Beziehung zwischen Datenverteilung und Modell nachzuvollziehen. Um diesen Sachverhalt zu verdeutlichen, folgt ein einfaches Beispiel:

Ein Münzwurf soll modelliert werden. Das Ergebnis von Münzwürfen ist mit der Wahrscheinlichkeitsverteilung $p(X)$ beschrieben wobei $x \in \left \{ \text{Kopf}, \text{Zahl} \right \}$. Um zu verdeutlichen, dass es sich bei $p(X)$ um die Datenverteilung handelt wird sie oft auch als $p_\text{data}(X)$ bezeichnet. \\
Das Modell, welches den Münzwurf modellieren soll wird $p_\theta(X)$ bezeichnet, wobei $\theta$ die Parameter dieses Modells darstellt. Das Ziel ist, dass Münzwürfe des Modells, nicht von tatsächlichen Münzwürfen unterschieden werden können. Einfacher ausgedrückt, müssen die Wahrscheinlichkeiten für Kopf und Zahl jeweils gleich sein: $p_\theta(X) \overset{!}{=} p_\text{data}(X)$. \\ 
Um dies zielgerichtet erfüllen zu können benötigen wir eine Metrik, welche angibt, wie gut das Modell die Datenverteilung approximiert. Hierfür kann beispielsweise die Distanz zwischen beiden Verteilungen genommen werden. Eine Möglichkeit diese zu Berechnen ist die KL-Divergenz $D_\text{KL}(p_\text{Data}(X)||p_\theta(X))$. Um eine solche Berechnung durchzuführen muss $p_\text{Data}(X)$ allerdings bekannt sein. Grundsätzlich ist dies allerdings für Datenverteilungen nicht möglich. \\
Die Minimierung der KL-Divergenz ist allerdings mathematisch äquivalent zur minimierung Kreuzentropie\footnote{
    Goodfellow, Bengio, Courville: Deep Learning, S. 75
    \cite{Goodfellow-et-al-2016}
}. 
Dies entspricht wiederum der maximierung der Log-Likelihood und ist somit als Trainingskriterium bei Modellen bekannt.
Damit $p_\theta(X)$ $p_\text{data}(X)$ erlernen kann, werden zunächst Beispiele benötigt. Dazu wird mehrfach eine Münze geworfen. Die Menge all dieser Ergebnisse sind die Trainingsdaten. Ein einzelner Münzwurf $x_n$ der Menge entspricht einem Sample, welches der Datenverteilung entnommen wurde $x_n \sim p_\text{data}(X)$. Nun kann über die Maximierung der Log-Likelihood normales Training vollzogen werden. \\
Wurde das Training abgschlossen stellt $p_\theta(X)$ nun eine Approximation für $p_\text{data}(X)$ dar. Um neue Münzwürfe zu erhalten, können wir sie nun also aus $p_\theta(X)$ entnehmen $x \sim p_\theta(X)$. Hierbei spricht man ebenfalls von Generierung oder Sampling.

Dieses Beispiel kann auf andere Anwendungsfälle übertragen werden. In der Bildysnthese entspricht die Datenverteilung beispielsweise einer Verteilung von Farbwerten auf $n \times m$ Pixeln, im Gegensatz zu einem Münzwurf auf Kopf oder Zahl.

% Autoencoder %%%%%%%%%%%%%%%%%%%%%%%%%%%%%%%%%%%%%%%%%%%%%%%%%%%%%%%%%%%%%%%%%
\subsection{Autoencoder}

Autoencoder (AE) sind streng genommen keine generativen Modelle, da sie lediglich rekonstruieren und keine neuen Daten erzeugen. Sie stellen allerdings die Grundlage für die Variational Autoencoder dar, welche generativ sind und im folgenden Unterabschnitt behandelt werden. \\ 
AEs haben als grundlegendes Ziel haben eine Eingabe in der dimensionalität zu reduzieren und anschließend wieder zur ursprünglichen Eingabe zu rekonstruieren. Man kann sie sich entsprechend als eine erlernte Variante von Kompressionsalgorithmen wie ZIP vorstellen. 
Zu diesem Zweck bestehen AEs aus zwei Komponenten\footnote{
    Die Encoder und Decoder Komponenten werden hier als $q$ und $p$ bezeichnet um die Verbindung zu Variational Autoencodern zu verdeutlichen. In vielen Publikationen werden sie allerdings $E$ und $D$ denotiert.    
}:
\begin{itemize}
    \item \textbf{Encoder $q_\phi(x)$}: \\
    Erhält als Eingabe Samples der Datenverteilung und komprimiert sie zu einem Code $h$.
    \item \textbf{Decoder $p_\theta(h)$}: \\
    Erhält als Eingabe den versteckten Zustand und rekonstruiert das ursprüngliche Sample der Datenverteilung möglichst verlustfrei. Diese Komponente modelliert die Datenverteilung auf basis des Codes.
\end{itemize}

Das Training eines solchen AEs ist sehr simpel. Man minimiere lediglich die Distanz zwischen Eingabedaten und Rekonstruktionen. Somit ergibt sich folgendes Optimierungsziel:

\begin{equation}
    \min L_{\phi, \theta}(x) = \mathbb E _{x \sim p_{data}} 
    \left [ 
        d(x, p_\theta(q_\phi(x)))
    \right ]
\end{equation}

% VAE %%%%%%%%%%%%%%%%%%%%%%%%%%%%%%%%%%%%%%%%%%%%%%%%%%%%%%%%%%%%%%%%%%%%%%%%%
\subsubsection{Variatonal Autoencoder}

Eine spezialisierte Variante der Autoencoder sind sogenannte Variational Autoencoder (VAE)\footnote{
    Kingma, Welling: Auto-Encoding Variational Bayes
    \cite{kingma2013auto}
}. Ihre Erweiterung der Autoencoder besteht in der Einführung einer Bedingung an den Code. Dieser kann in im ursprünglichen AE ein praktisch beliebiger Vektor sein, solange er bei der Entschlüsselung von Vorteil ist. Je nach Anwendungsfall kann es allerdings hilfreich sein, wenn der Code selbst auch eine semantisch Aussagekräftige Darstellung der Eingabe ist. Dies kann den Code, je nach Anwendungsfall, zu einem geringerdimensionalen Substitut für die Ursprungsdaten machen. Hierbei ist allerdings wichtig zu beachten, dass es sich hierbei in aller Regel nicht um menschenverständliche Eigenschaften handelt. \\
Zu diesem Zweck wird der latente Ergebnisraum des Encoders mit einer beliebigen Wahrscheinlichkeitsverteilung $p(z)$ kombiniert. Eingabedaten werden somit also einen latenten Raum $Z$ abgebildet, welcher grob $p(z)$ entspricht. \\
Die zugrundeliegende Aussage die somit getroffen wird ist, dass jede Komponente des Codes nicht nur eine Kompression der Eingabe ist, sondern auch selbst einer Wahrscheinlichkeitsverteilung folgt. Hierbei wird angenommen, dass Merkmale der Datenverteilung, die beispielsweise einer Normalverteilung folgen, eine strukturierte und nützliche Darstellung der Eingabedaten ermöglichen. Die so entstehende Verteilung hat die folgende Form:
\begin{equation}
    p_\theta(x) = \int_z p_\theta(x, z) dz
\end{equation}
Diese Anpassung hat zur Folge, dass das Training eines VAE ein neues Optimierungsziel benötigt. Der erste Gedanke ist, die Log-Likelihood der Verteilung $p_\theta(X)$ zu maximieren. Dies ist aufgrund des Integrals über den latenten Raum nicht effizient möglich. Dieses Problem wird gelöst, indem eine zweite Quantität maximiert wird, welche garantiert kleiner als die Log-Likelihood ist. Diese Quantität ist das Optimierungsziel und wird allgemein die \textit{Evidence Lower Bound} (ELBO) genannt. Sie ist gegeben als:
\begin{equation}
    \mathcal \max L_{\phi, \theta}(x) = \mathbb E_{z \sim q_{\phi}(z|x)}
    \left [
        \log p_\theta(x|z)
    \right ]
    - D_\text{KL} (\log q_{\phi}(z|x) || p(z))
\end{equation}
$p(z)$ ist hierbei die, für die Verteilung der Merkmale angenommene Verteilung. In der Regel handelt es sich hierbei um eine Standardnormalverteilung $\mathcal N(0; I)$. Die Herleitung der ELBO ausgehend von der Log-Likelihood ist sehr komplex und dem Verständnis nicht weiter zutragend. Eine Betrachtung der einzelnen Therme bietet allerdings Aufschluss über die Bedeutung der ELBO:
\begin{itemize}
    \item $\mathbb E_{z \sim q_{\phi}(z|x)}
        \left [
            \log p_\theta(x|z)
        \right ]$: \\
    Hierbei handelt es sich um die Log-Likelihood der Eingabedaten unter der Bedingung der jeweiligen latenten Repräsentation. Einfacher ausgedrückt stellt dies die Übereinstimmung der Eingabe und der Rekonstruktion sicher. Somit ist dies lediglich das Optimierungsziel des ursprünglichen AE.
    \item $-D_\text{KL} (\log q_{\phi}(z|x) || p(z))$: \\
    Diese KL-Divergenz stellt eine Regularisierungstherm dar. Sie fordert, dass die Verteilung $q_{\phi}(z|x)$, sprich der latente Ergebnisraum des Encoders, der angenommenen zugrundeliegenden Verteilung $p(z)$ entspricht.
\end{itemize}
Tatsächlich wurde also, gegenüber des ursprünglichen Optimierungsziels nur ein Regularisierungstherm hinzugefügt.



% GAN %%%%%%%%%%%%%%%%%%%%%%%%%%%%%%%%%%%%%%%%%%%%%%%%%%%%%%%%%%%%%%%%%%%%%%%%%
\subsection{Generative Adversarial Networks}

Maximierung der Log-Likelihood ist nicht die einzige Methode eine Datenverteilung zu erlernen. Tatsächlich schränkt sie häufig die Komplexität der zugrundeliegenden Funktionen ein, da sie von Modellen fordert, dass sie als Wahrscheinlichkeitsverteilungen formuliert werden, welche somit die damit verbundenen mathematischen Invarianten erfüllen müssen. Dies kann die Qualität der Ergebnisse beeinträchtigen. Insbesondere bei Bilddaten kommt hinzu, dass die Log-Likelihood nicht unbedingt ein gutes Maß für visuelle Qualität ist.\\
Generative Adversarial Networks (GANs)\footnote{
    Goodfellow et al.: Generative Adversarial Nets
    \cite{goodfellow2014generativeadversarialnetworks}
} erlernen die Datenverteilung auf eine andere Weise. In ihnen kommen zwei Komponenten zum Einsatz der \textit{Diskriminator} und der \textit{Generator} welche zwei gegensätzliche Ziele verfolgen:
\begin{itemize}
    \item \textbf{Diskriminator $D_\phi$}: \\
    Der Diskriminator ist, wie der Name vermuten lässt, kein generatives Modell. Er hat als Ziel zwischen den Ausgaben des Generators und Samples der Datenverteilung zu unterscheiden.
    \item \textbf{Generator $G_\theta$}: \\
    Das Ziel des Generators ist das Erzeugen von Datenpunkten andhand eine einfachen Ausgangsverteilung $p_z$, die vom Diskriminator nicht von echten Daten unterschieden werden können. 
\end{itemize}
Die Idee hinter diesen gegeneinander gerichteten Komponenten ist, dass somit Daten generiert werden, die möglichst ähnlich zu realen Daten sind. Das Optimierungsziel entspricht somit dieser Form\footnote{
    Goodfellow et al.: Generative Adversarial Nets, S. 3
    \cite{goodfellow2014generativeadversarialnetworks}
}:
\begin{equation}
    \min_{G_\theta}\max_{D_\phi}V(G_\theta, D_\phi)
    = \mathbb E_{x \sim p_\text{data}}[\log D_\phi(x)] 
    + \mathbb E_{z \sim p_z}[\log (1 - D_\phi(G_\theta(z)))]
\end{equation}
Die Gegensätzlichkeit ist hier durch das Minimax ausgedrückt, bei welchem der $V$ respektiv zum Generator minimiert und zum Diskriminator maximiert wird. Hervorzuheben ist, dass der Generator über das Täuschen des Diskriminators traininiert wird. \\
Diese Herangehensweise ist zwar gut um Erzeugen von hochqualitativen Samples, allerdings erweist sie in der Praxis viele Probleme, die nur schwierig zu bewältigen sind. Ein prominentes dieser Probleme ist der sogenannte \textit{Mode Collapse}. Hierbei handelt es sich um einen Zustand, bei dem der Generator nur einen kleinen Bereich der Datenverteilung erlernt. Dies ist ein Umstand der auf das Trainingskriterium des Generators zurückzuführen ist. Hierbei ist, im Gegensatz zur Maximierung der Log-Likelihood, nicht sichergestellt, dass die gesamte Datenverteilung betrachtet wird. Es kann ausreichen nur einen kleinen bereich zu erlernen, welcher aber vom Diskriminator als real eingestuft wird.\footnote{
    Thanh-Tung, Tran: Mode Collapse in Generative Adversarial Networks, S. TODO
    \cite{thanhtung2020catastrophicforgettingmodecollapse}
}

\subsubsection{VAE-GAN}

Insbesondere bei Bilddaten können Log-Likelihood basierte Modelle qualitative Mängel aufweisen. Dies ist darin begründet das Optimierungsziele zum Optimieren der Log-Likelihood wie beispielsweise die Kreuzentropie nicht unbedingt gute Kriterien für visuelle Ähnlichkeit zur Datenverteilung sind. So kommt es bei VAEs unter anderem oft zu verwaschenen Rekonstruktionen. \\
VAE-GANs haben als Ziel, die visuelle Qualität von Ergebnissen von VAEs zu verbessern. Sie tun dies indem sie, wie ihr Name vermuten lässt, VAEs mit GANs kombinieren. Konkret wird hierbei der Decoder des VAEs als Generator eines GANs angesehen. Somit müssen die aus dem Decoder resultierenden Rekonstruktionen einen Diskriminator täuschen, welcher die visuelle Qualität sicherstellen soll. Das Trainingsziel ist hierbei ebenfalls eine Kombination beider Ansätze\footnote{
    Larsen et al.: Autoencoding using a learned similarity metric, S. 2
    \cite{larsen2016autoencoding}
}:
\begin{equation}
    L_\text{VAE-GAN}
    =  L_\text{prior} +  L_{D_l\text{ like}} +  L_\text{GAN}
\end{equation}
Wobei $ L_\text{prior}$ und $ L_\text{GAN}$ bereits vorgestellt wurden und $ L_{D_l\text{ like}}$ eine Abwandlung des ersten Therms der ELBO darstellt. Diese beruht nun auf den Aktivierungen der $l$-ten Schicht des Diskriminators des GAN. 
\begin{equation}
    L_\text{prior} 
    = D_\text{KL} (\log q_{\phi_\text{VAE}}(z|x) || p(z))
\end{equation}
\begin{equation}
    L_\text{GAN} = \mathbb E_{x \sim p_\text{data}}[\log D_{\phi_\text{GAN}}(x)] 
    + \mathbb E_{z \sim p_z}
    [\log (1 - D_{\phi_\text{GAN}}(G_{\theta_\text{GAN}}(z)))]
\end{equation}
\begin{equation}
    L_{D_l\text{ like}} = 
    - \mathbb E_{z \sim q_{\phi_\text{VAE}}(z|x)}
    \left [
        \log p_{\theta_\text{VAE}}(D_{l, \phi_\text{GAN}}(x)|z)
    \right ]
\end{equation}

In der Praxis wird allerdings oft anstelle von $L_{D_l\text{ like}}$ der ursprüngliche Rekonstruktionsfehler $- \mathbb E_{z \sim q_{\phi}(z|x)}
    \left [
        \log p_\theta(x|z)
    \right ]$ von VAEs angewandt. \\
Diese Variante von VAEs ist in der Lage deutlich ähnlicher scheinende Rekonstruktionen zu erzeugen, als es einfache VAEs können.


\subsubsection{Visual Similarity Metrics}



% SUPPORTING MODELS %%%%%%%%%%%%%%%%%%%%%%%%%%%%%%%%%%%%%%%%%%%%%%%%%%%%%%%%%%%

\subsection{U-Net}
\label{subsec:Unet}

Eine in der Image-To-Image Domäne weitverbreitete Modellarchitektur ist das  sogenannte U-Net\footnote{
    Ronneberger, Fischer, Brox: Convolutional Networks for Image Segmentation
    \cite{ronneberger2015unetconvolutionalnetworksbiomedical}
}. Ursprünglich wurde Sie für Bildsegmentierung, also dem Kennzeichnen zusammenhängender oder relevanter Bildelemente, entwickelt. Mitlerweile findet sie jedoch in vielen weiteren Bereichen, welche Bilddaten zu Bilddaten verarbeiten verwendung. \\
Ihr Aufbau wird in ihrer Vorstellung wie folgt dargelegt\footnote{
    Vgl. Ronneberger, Fischer, Brox: Convolutional Networks for Image Segmentation, S. 4
    \cite{ronneberger2015unetconvolutionalnetworksbiomedical}
}. Sie besteht aus einem Kompressions- und einem Expansionsarm. der Kompressionsarm hat hierbei die übliche Struktur eines Convolutional Networks. Er besteht aus wiederholter Anwendung von 3x3 Konvolutionen und folgendem Downsampling, welches die Dimensionen halbiert und die Anzahl der Featurekanäle verdoppelt. \\
Der Expansionsarm kehrt die Kompression um und besteht somit aus Upsampling-Blöcken, welche die Anzahl der Featurekanäle halbieren. Diesem Upsampling folgen wiederum wiederholte 3x3 Konvolutionen. Hierbei werden in der ursprünglichen Umsetzung die komprimierten Features des Kompressionsarms mit den expandierten konkatteniert. In bestimmten Anwendungsfällen kann dies jedoch ausgelassen werden. 





\subsection{Transformer}

Der Transformer\footnote{
    Vaswani et al.: Attention is All You Need
    \cite{vaswani2023attentionneed}
} ist eine Modellarchitektur, welche ursprünglich für den Bereich der Generierung von natürlicher Sprache entworfen wurde. Sie zeichnen sich insbesondere durch ihre Fähigkeit aus, sequenzielle Daten in einem einzigen Zeitschritt kontextualisiert zu verarbeiten. Mitlerweile findet sie allerdings auch in vielen weiteren Gebieten des Deep-Learning Verwendung, so auch beispielsweise in der Bildsynthese in Form von Vision Transformern\footnote{
    Dosovitskiy et al.: An Image is Worth 16x16 Words 
    \cite{dosovitskiy2021imageworth16x16words}
}. \\ 
Der Transformer folgt dem Encoder-Decoder Schema. Dabei dient, vereinfacht ausgedrückt, der Encoder zur Verarbeitung der Eingabe zu einem Kontext. Der Decoder hingegen generiert anhand dieses Kontextes und der bisher generierten Sequenz autoregressiv die Ausgabe. Das bedeutet, dass die Ausgabe Komponente um Komponente, auf Basis des bisherigen Ergbnisses, synthetisiert wird. \\ 
Die Daten, die hierbei verarbeitet werden sind grundsätzlich sequenziel, ihre Position innerhalb der Sequenz ist also für ihre Interpretation relevant. \\
Die grobe Funktionsweise ist dabei wie Folgt. Die Eingabedaten müssen zunächst für den Encoder angepasst werden. Dies kann das gegebenfalls das Embedden des Eingabevektors beinhalten, in jedem Fall aber die Integration eines Positional Embeddings. Diese weist jeder Komponente des Eingabvektors einen eindeutigen Wert anhand der Position dieser Komponente in der zu verarbeitenden Sequenz. Dieses Positional Encoding beruht dabei üblicherweise auf einer Kosinusfunktion, kann aber auch erlernt werden. \\
Dieser um die Positionsinformationen angereicherte Vektor kann nun vom Encoder, zu einem sogenannten Contextual- oder Hidden-Embedding, verarbeitet werden. Dies ist ein Vektor, welcher grundsätzlich dem Eingangsvektor entspricht. Jede Komponente dieses Vektors ist dabei, stark vereinfacht ausgedrückt, eine kontextualisierte Repräsentation der ursprünglichen Komponente. Das heißt, sie beinhaltet Informationen über ihre Bedeutung innerhalb der jeweiligen Sequenz, sowie gegebenenfalls im breiteren Kontext der Domäne des Modells. \\
Anhand dieses Hidden-Embeddings wird nun Komponente für Komponente die Ausgabesequenz generiert. Dabei wird in jedem Schritt das bis zu diesem Zeitpunkt generierte Ergebnis mit dem Contextual-Embedding in Kontext gesetzt, was letztendlich die Grundlage für die Auswahl der wahrscheinlich als nächstes folgenden Komponente der Sequenz ist.

\subsubsection{Attention}

Der Erfolg von Transformern beruht maßgeblich auf ihren Attention-Blöcken, diese Erlauben es einen, oder gegebenfalls auch zwei, Sequenzvektoren zu sich selbst, oder einander in Kontext zu setzen und diese Information in ihrer Ausgabe festzuhalten. \\
Diese basieren dabei auf der der sogenannten \textit{Self-Attention}, welche einen sogenannten Attention-Vektor als Ergebnis hat. Jede Komponente dieses Vektors wird berechnet indem alle Komponenten eines Eingabevektors linear abgebildet werden:
\begin{equation}
    y_i = \sum_{j=1}^n w_{ji}x_j
\end{equation}
Wobei $x$ der Eingabevektor, $y$ der Attentionvektor und $W$ die Gewichtsmatrix für die lineare Abbildung sind. Hierbei ist die Invariante, dass die Summe der Spaltenvektoren von $W$ gleich eins ist, zu erfüllen. Selfattention kann somit als Wertung der Wichtigkeit aller Komponenten von $x$ gemäß verschiedener diskreter Warhscheinlichkeitsverteilungen verstanden werden. Mit anderen Worten wird somit bestimmt, wieviel Aufmerksamkeit einer Komponente zukommen soll. \\
Dieses Prinzip wird in Attention-Blöcken grundlegend durch die \textit{Scaled Dot Product Attention} umgesetzt. Diese ist wie folgt definiert\footnote{
    Vaswani et al.: Attention is All You Need, S. 4
    \cite{vaswani2023attentionneed}
}:
\begin{equation}
    \text{Attention}(Q, K, V) = \text{softmax}
    \left (
        \frac {QK^T} {\sqrt{d_k}}
    \right ) V
\end{equation}
\begin{equation}
    \text{softmax}(z) = \frac{e^z}{\sum_{j=1}^K e^{z_j}}
\end{equation}
$Q$, $K$ und $V$ sind hierbei Ergebnisse der Multiplikation eines Eingabevektors mit den korrespondierenden erlernten Gewichtsmatrizen $W^Q$, $W^K$ und $W^V$. $d_k$ ist die Dimension von $Q$ und $K$. \\
Man nennt $Q$, $K$ und $V$ auch \textit{Query}, \textit{Key} und \textit{Value}, was ihre Bedeutung veranschaulicht. Zunächst werden Query und Key werden miteinander multipliziert, der sich hieraus ergebende Wert soll angeben, wie relevant der jeweilige Key für die jeweilige Query ist. Dies entspricht der Definition der Self-Attention und die umschließende Softmax-Funktion, stellt die Einhaltung der Invariante sicher. Diese, vereinfacht ausgedrückt, gewichteten Relevanzwerte, werden anschließend mit dem Value verrechnet, welcher der Repräsentation der kontextuellen Bedeutung der jeweiligen Komponente des Eingangsvektors entspricht. \\
Sollen zwei Vektoren zu einander in beziehung gesetzt werden, so wird einer der beiden für $Q$ und der andere für $K$ und $V$ verwendet. Man spricht in diesem Fall auch von \textit{Cross Attention}. \\
Diese Berechnung kann auch mehrfach mit unterschiedlichen $W^Q$, $W^K$ und $W^V$ durchgeführt werden, um unterschiedliche Interpretationsmöglichkeiten der Sequenz abbilden zu können. Dies wird \textit{Multi-Head Attention} genannt.

% DDPM %%%%%%%%%%%%%%%%%%%%%%%%%%%%%%%%%%%%%%%%%%%%%%%%%%%%%%%%%%%%%%%%%%%%%%%%
\subsection{Diffusionmodelle}
\label{subsec:Grundlagen_DMs}

Wie in vorherigen Abschnitten bereits erläutert, sind Log-Likelihood basierte Modelle in strenge mathematische Bedingungen gebunden. Dies schränkt ihre mögliche Komplexität ein. Modelle welche andere Kriterien optimieren, wie GANs erlauben zwar höhere Komplexität in ihren Funktionen, haben allerdings praktische Fallstricke wie das Mode-Collapse Problem. Somit stellt sich die Frage, ob es eine Klasse von Modellen gibt, welche die Log-Likelihood als Optimierungsziel haben und gleichzeitig beliebig komplexe Funktionen erlauben. \\
Eine Antwort hierauf stellen Diffusions Modelle (DM)\footnote{
    Dickstein et al.: Diffusion Models
    \cite{pmlr-v37-sohl-dickstein15}
}. Die Kernidee der Generierung dieser Klasse von Modellen ist dabei das, über viele Zeitschritte hinweg, iterative Herauskristallisieren von Informationen aus einem ursprünglichen Rauschbild. Intuitiv kann dies damit begründet werden, dass jeder einzelne Schritt hierbei eine Verhältnismäßig einfache Aufgabe ist, da das Ergebnis jedes Schrittes nur eine kleine Änderung des verrauschten Ausgangszustands darstellt. Alle Schritte zusammen jedoch ergeben ein sehr tiefes und mächtiges Modell. Vergleichbar ist dies mit einem Künstler welcher, beginnend mit einer Grundierung, Schicht für Schicht weitere Strukturen und Details auf eine Leinwand aufträgt. \\
Die Formulierung eines Diffusionsmodells ist in zwei Prozesse aufgeteilt, wobei $t \in [0,T]$ den jeweiligen Zeitschritt angibt. Hierbei ist eine gängige Konvention $T=1000$: 
\begin{itemize}
    \item \textbf{Vorwärtsprozess} $q(x_{1:T}|x_0)$: \\
    Das Ziel des Vorwärtsprozesses ist das iterative Einführen von zunehmenden, in der Regel gaußschen Rauschen in die Eingabedaten $x_0$. Die Daten zum letzten Zeitschritt $x_T$ stellen hierbei reines Rauschen dar, in welchem die Informationen von $x_0$ gänzlich zerstört wurden. Dieser Prozess ist der Grund für die Namensgebung von Diffusionsmodellen, da er dem Vorgang der Diffusion in der Physik ähnelt. Definiert ist er wie Folgt:
    \begin{equation}
        q(x_{1:T}|x_0) := \prod_{t=1}^T q(x_t | x_{t-1}) 
    \end{equation}
    \begin{equation}
        q(x_t|x_{t-1}) :=  \mathcal N(x_t; \sqrt{1-\beta_t}x_{t-1}, \beta_t I)
    \end{equation}
    $\beta_t$ ist dabei ein zeitabhängiger Wert, welcher die Stärke des gaußschen Rauschens des jeweiligen Zeitschritts bestimmt. Hierfür existieren mehrere Strategien, wobei eine einfache, positive lineare Funktion in Abhängigkeit von $t$ bereits ausreicht. Diese Strategien werden auch Noise-Schedules genannt. Ein hervorzuhebenes Detail ist, dass $q$ vollständig definiert ist und somit keinerlei zu trainierende Parameter enthält. \\
    $q(x_{1:T}|x_0)$ lässt sich zudem noch soweit vereinfachen, dass jedes $q(x_t|x_0)$ in nur einem einzigen Zeitschritt ermitteln lässt\footnote{
        Ho, Jain, Abbeel: Denoising Diffusion Probabilistic Models, S. 2
        \cite{ho2020denoisingdiffusionprobabilisticmodels}
    }:
    \begin{equation}
        q(x_t|x_0) :=  
        \mathcal N(x_t; \sqrt{1-\bar\alpha_t}x_0, (1 - \bar \alpha_t) I)
    \end{equation}
    Mit $\alpha_t := 1 - \beta_t$ und $\bar\alpha_t := \prod_{s=1}^t \alpha_s$.

    \item \textbf{Rückwärtsprozess} $p_\theta(x_{0:T})$: \\
    Der Rückwärtsprozess versucht, dem Namen entsprechend, die im Vorwärtsprozess getätigten Diffusionsschritte umzukehren - also aus dem Rauschen $x_T$ die Ursprungsdaten $x_0$ iterativ zu rekonstruieren. Entsprechend ist auch die mathematische Formulierung:
    \begin{equation}
        p_\theta(x_{0:T}) := p(x_T) \prod_{t=1}^T p_\theta(x_{t-1} | x_{t}) 
    \end{equation}
    \begin{equation}
        p_\theta(x_{t-1} | x_{t})  :=  
        \mathcal N(x_{t-1}; \mu_\theta(x_{t}, t), \Sigma_\theta(x_{t}, t))
    \end{equation}
    Die Parameter des Rauschens $\mu$ und $\Sigma$ werden hierbei in Abhähnigkeit zum aktuellen und dem Ergebnis des jeweils vorherigen Zeitschritts durch neuronale Netze $\mu_\theta$ und $\Sigma_\theta$ abgebildet. Hierbei kommen nun erlernte Parameter $\theta$ zum Einsatz, da die Rekonstruktion eine deutlich kompliziertere Funktion darstellt, als die Diffusion. Ausgenommen ist hierbei $p(x_T)$, da dies ein vollständiges Rauschbild ist und somit $\mathcal N$ entspricht, also keine Parameter benötigt.
\end{itemize}
Die Notation dieser Prozesse ähnelt den Komponenten eines Autoencoder. Tatsächlich können Diffusionsmodelle als Variational Autoencoder verstanden werden, bei welchen die latente Abbildung einer Standardnormalverteilung entspricht\footnote{
    Ho, Jain, Abbeel: Denoising Diffusion Probabilistic Models, S. 3
    \cite{ho2020denoisingdiffusionprobabilisticmodels}
}.
Dementsprechend vergleichbar ist ebenfalls das Optimierungsziel, welches eine untere Grenze der Log-Likelihood, auch Variational Lower Bound (VLB) genannt, ist. Diese setzt sich aus drei grundlegenden Thermen, welche in Abhängigkeit zu $t$ stehen, zusammen\footnote{
    Dhariwal, Nichol: Diffusion Models Beat Gans on Image Synthesis, S. 19
    \cite{dhariwal2021diffusionmodelsbeatgans}
}:
\begin{equation}
    \min L_\text{vlb} := L_0 + L_{1:T-1} + L_T 
\end{equation}
Wobei die einzelnen hergeleiteten Therme hierbei folgend definiert sind:
\begin{equation}
    L_0 := -\log p_\theta(x_0|x_1)
\end{equation}
\begin{equation}
    L_{1:T-1} := D_\text{KL}(q(x_{t-1}|x_t,x_0)||p_\theta(x_{t-1},x_t))
\end{equation}
\begin{equation}
    L_T := D_\text{KL}(q(x_T|x_0)||p(x_T))
\end{equation}
Diese Definition erscheint auf den ersten Blick nur schwer verständlich, bei genauerer Betrachtung jedoch lässt sich die Bedeutung relativ einfach zusammenfassen. Alle drei bilden jeweils die Distanz der korrespondierenden Vorwärts- und Rückwärts-Verteilungen ab. Im allerersten Schritt, also $t=0$, sind die Eingabedaten noch unverrauscht. Somit müssen hier noch nicht die Normalverteilungen verglichen werden sondern lediglich die negative Log-Likelihood. Im letzten Zeitschritt werden reine Normalverteilungen verglichen, es benötigt also keine Parametrisierung $\theta$. Für alle schritte dazwischen müssen die jeweiligen Normalverteilungen mittels KL-Divergenz verglichen werden. Die zusätzliche Konditionierung des Vorwärtsprozesses durch $x_0$ ist hierbei notwendig, damit die Berechnung in polynomialer Zeit durchgeführt werden kann\footnote{
    Ho, Jain, Abbeel: Denoising Diffusion Probabilistic Models, S. 3
    \cite{ho2020denoisingdiffusionprobabilisticmodels}
}. \\
Das Training erfolgt dabei pro Zeitschritt, welche üblicherweise für jeden Optimierungsschritt zufällig ausgewählt werden. Beim Sampling werden hingegen alle Zeitschritte in einer Markow-Kette, beginnend bei $t=T$ durchlaufen. 


%%%%%%%%%%%%%%%%%%%%%%%%%%%%%%%%%%%%%%%%%%%%%%%%%%%%%%%%%%%%%%%%%%%%%%%%%%%%%%%
\subsubsection{Denoising Diffusion Probabilistic Models}

Die vorherige Definition von Diffusionsmodellen im Allgemeinen bietet bereits ein Verständnis für die Leistungsfähigkeit dieser Klasse von Modellen. Das Prinzip des iterativen Erkennens von Strukturen in Rauschen ist bereits intuitiv leicht zu greifen. Eine besondere Formulierung von Diffusionsmodellen - Denoising Diffusion Probabilistic Models\footnote{
    Ho, Jain, Abbeel: Denoising Diffusion Probabilistic Models
    \cite{ho2020denoisingdiffusionprobabilisticmodels}
} (DDPMs) - macht die Begründung des Rauschens noch leichter zu erfassen. \\
Die grundlegende Umformulierung ist nun, dass anstelle des Erkennens der Struktur in einem Rauschbild, das Rauschen in einer verrauschten Struktur erkannt und iterativ entfernt wird. Auf den ersten Blick mag bereits ersichtlich sein, dass diese zwei Herangehensweisen grundsätzlich äquivalent sind, dieser Perspektivwechsel erlaubt allerdings neue Erkenntnisse. \\
Zuerst wird auf diese Weise der Zusammenhang zu sogenannten \textit{Score Matching Modellen} (SMM) offensichtlich. Diese Klasse von generativen Modellen erlernt die Datenverteilung nicht direkt, sondern ihren Score, also den Gradienten der Log-Likelihood. Auf diese Weise werden Eingabedaten also auf einen Vektor abgebildet, welche anzeigt wie die Daten verändert werden müssten, damit sie unter der Ursprungsverteilung wahrscheinlicher werden. Somit wird also indirekt ebenfalls die Datenverteilung erlernt. Der Samplingprozess unter einem solchen Modell ist das iterative Anpassen eines ursprünglichen Verrauschten Vektors, zu einem Sample, welches eine hohe Log-Likelihood unter der erlernten Verteilung hat. Dieser Prozess entspricht exakt nun exakt dem Rückwärtsprozess eines Diffusionsmodells. Nun kann die Frage aufkommen, wie wird in einem Diffusionsmodell nun aber der Score erlernt, wenn lediglich die Normalverteilungen des Vor- und Rückwärtsprozess angeglichen werden. Hier ist nun der Perspektivwechsel zum Erkennen des Rauschens relevant. Zu erlernen welches Rauschen in Daten vorhanden ist, ist äquivalent zum Erlernen des Scores. Um dies zu begreifen, muss sich vor Auge geführt werden, dass das Verrauschen eines einzelnen Datenpunktes bedeutet, dass er von seinem ursprünglichen Wert verschoben wird. Rauschen zu entfernen bedeutet somit, zu erlernen in welche Richtung und um wie viel ein Wert verschoben wurde. Mit anderen Worten also: Den Gradienten zu erlernen, welcher zum maximal wahrscheinlichen Ausgangswert weist. Was eben genau die Definition des Scores ist. Von Vorteil bei dieser Art von Modellen ist, dass sie gleichzeitig die tatsächliche Datenverteilung erlernern, wenn auch indirekt, und trotzdem nicht an die Invarianten einer WDF gebunden sind, da sie eben nur den Score erlernen. \\
Weiter kann so, unter der Annahme, dass die Kovarianzmatrix der Normalverteilungen konstant $\beta_t$ ist (Diese Annahme wird im folgenden Unterabschnitt kritisch hinterfragt), ein neues, deutlich simpleres Optimierungsziel formuliert werden\footnote{
    Ho, Jain, Abbeel: Denoising Diffusion Probabilistic Models, S. 5
    \cite{ho2020denoisingdiffusionprobabilisticmodels}
}: 
\begin{equation}
    \min L_\text{simple} = \mathbb E_{\epsilon \sim \mathcal N}
    \left [
        \| \epsilon - \epsilon_\theta(x_{t}, t) \|^2
    \right ]
\end{equation}
Hierbei erlernt das neurnale Netz $\epsilon_\theta$ nun nicht mehr Erwahrtungswert und Standardverteilung der zugrundeliegenden Normalverteilung, sondern das Rauschen $\epsilon$ selbst. 

%%%%%%%%%%%%%%%%%%%%%%%%%%%%%%%%%%%%%%%%%%%%%%%%%%%%%%%%%%%%%%%%%%%%%%%%%%%%%%%
\subsubsection{Improved DDPMs}

In der Praxis hat sich die Annahme, dass die Kovarianzmatrizen der Normalverteilungen konstant sind als zu vereinfachend herausgestellt. Bessere Ergebnisse können durch das Erlernen von $\Sigma$ erzielt werden. Allerdings erhöht dies in der Praxis die Varianz der Trainingsverluste. Eine erfolgreiche Lösung wird Nichol und Dhariwal\footnote{
    Nichol, Dhariwal: Improved Denoising Diffusion Probabilistic Models, S. 4
    \cite{nichol2021improveddenoisingdiffusionprobabilistic}
} vorgeschlagen. Sie kombinieren die Optimierungsziele $L_\text{simple}$ und $L_\text{vlb}$:
\begin{equation}
    L_\text{hybrid} := L_\text{simple} + \lambda L_\text{vlb}
\end{equation}
$\lambda$ ist hierbei ein Gewicht, welches dafür sorgen soll, dass $L_\text{vlb}$ $L_\text{simple}$ nicht überwältigt. Ein wichtiges Detail ist, dass der valide Bereich für $\Sigma$ sehr klein ist. Deshalb wird das Ergebnis $v$ des neuronalen Netzes für $\Sigma$ zunächst noch durch $\beta_t$ und $\tilde \beta_t$ begrenzt:
\begin{equation}
    \Sigma_\theta(x_t, t) = e^{v \log \beta_t + (1 - v) \log \tilde \beta_t}
\end{equation}
\begin{equation}
    \tilde \beta_t = \frac{1-\bar \alpha_{t-1}} {1-\bar \alpha_{t}} \beta_t
\end{equation}
Die Autoren schlagen ebenfalls eine neue nicht-lineare Noise-Schedule vor welche auf dem Cosinus basiert: 
\begin{equation}
    \beta_t = \cos \left ( 
        \frac{t/T+s}{1+s} \frac{\pi}{2}
    \right )^2
\end{equation}
Diese soll dafür sorgen, dass Informationen in eingangsdaten langsamer zerstört werden als bei einer linearen Funktion und somit in jeder Trainingsschritt aussagekräftiger wird. $s$ ist von den Autoren auf 0.008 belegt worden.

%%%%%%%%%%%%%%%%%%%%%%%%%%%%%%%%%%%%%%%%%%%%%%%%%%%%%%%%%%%%%%%%%%%%%%%%%%%%%%%
\subsubsection{Konditionierung}

Diffusionsmodelle wie sie soweit vorgestellt wurden sind unkonditionel. Es gibt somit also noch keine Möglichkeit über ein Kontrollsignal, wie beispielsweise Klassen, die Generation zu steuern. \\
Um dies zu ermöglichen muss das Signal zunächst in das Modell integriert werden. Hierfür existieren viele Ansätze, welche alle gemein haben, dass das Signal üblicherweise vor der Integration embeddet wird. Das bedeutet, dass es zunächst auf einen Vektor festgelegter größe abgebildet wird. Diese abbildung kann fest definiert, oder erlernt werden. Einige gängige Varianten der Integration solcher Embeddings werden folgend aufgelistet:
\begin {itemize}
    \item \textbf{Konkatenation mit Eingangsdaten}: \\
    Die wahrscheinlich simpelste Art der Integration ist die Konkatenation des Embeddings mit den Eingabedaten.
    \item \textbf{Addition mit Zeitembedding}: \\
    Eine ebenfalls sehr simple alternative ist die Addition des Embeddings mit dem Zeitschritt-Embedding. Hierbei muss sichergestellt werden, dass beide Vektoren die gleiche Dimensionalität haben.
    \item \textbf{Cross Attention}: \\
    Eine deutlich komplexere Variante ist die Integration mittels Attention. Hierbei werden während der Verarbeitung der Eingabedaten gegebenenfalls mehrfach Cross-Attention Blöcke verwendet, wobei das Embedding für Key und Value genutzt werden. Dieser Ansatz eignet sich vor allem für sehr komplexe Kontrollsignale, wie beispielsweise Text.
    \item \textbf{Adaptive Group Normalization} (AdaGN) \footnote{
        Vgl. Dhariwal, Nichol: Diffusion Models Beat Gans on Image Synthesis, S. 6
        \cite{NEURIPS2021_49ad23d1}
    }: \\
    Eine weitere Methode sind AdaGN-Blöcke. Diese erlernen eine lineare Abbildung von Zeit- und Konditionsembedding, dabei können diese Embeddings zuvor addiert oder konkatteniert werden. Sie werden ebenfalls meist mehrfach verwendet. Definitiert ist AdaGN wie folgt:
    \begin{equation}
        \text{AdaGN}(h,y) = y_s \text{GroupNorm}(h) + y_b
    \end{equation}
    Wobei $h$ die aktuellen Aktivierungen und $y=[y_s, y_b]$ das aufgeteilte Ergebnis einer erlernten linearen Abbildung von Zeit und Kondition ist. \\
    Eine etwas abgewandelte Methode ist Adaptive Layer Normalization (AdaLN), welche lediglich die Group Normierung durch eine Layer Normierung ersetzt.

    
\end {itemize}  
Nach Integration in das Modell, müssen ebenfalls die Algorithmen für Training und Sampling angepasst werden. Die gängigste Art und Weise hierfür stellt die \ac{CFG}\footnote{
    Ho, Salimans: Classifier-Free Diffusion Guidance
    \cite{ho2022classifierfreediffusionguidance}
} dar. Durch sie ist es möglich das Diffusionsmodell zu Konditionieren, ohne dass ein Klassifikationsmodell verwendet werden muss, wie es zuvor nötig war\footnote{
    Vgl. Dhariwal, Nichol: Diffusion Models Beat Gans on Image Synthesis, S. 6 ff.
    \cite{NEURIPS2021_49ad23d1}
}. 
Hierbei wird zuallererst ein Null-Embedding $\diameter$ definiert, welches die Abwesenheit eines Kontrollsignals abbildet. Anschließend müssen folgende geringfügige Anpassung für Training und Sampling vollzogen werden:
\begin{enumerate}
    \item \textbf{Training}:\\
    Beim Training wird nun die tatsächliche Kondition mit einer definierten Wahrscheinlichkeit $p$ durch $\diameter$ ersetzt. $p$ ist dabei eher gering, beispielsweise $p=0.1$
    \item \textbf{Sampling}:\\
    Für das Sampling wird jeder Zeitschritt zweimal durchgeführt. Einer davon nutzt dabei die Kondition $c$, der Andere hingegen $\diameter$. Anschließend wird zwischen beiden Ergebnissen mit einem Gewicht $\lambda$ linear interpoliert:
    \begin{equation}
        \tilde \epsilon_t(z_t, c) = (1+w)\epsilon_\theta(z_t, c)
        - w\epsilon_\theta(z_t, \diameter)
    \end{equation}
\end{enumerate}

%%%%%%%%%%%%%%%%%%%%%%%%%%%%%%%%%%%%%%%%%%%%%%%%%%%%%%%%%%%%%%%%%%%%%%%%%%%%%%%
\subsubsection{Image To Image}

Durch die Markow-Kette des Rückwärtsprozesses beim Samplings in einem \ac{DM} ergibt sich die Möglichkeit diesen Prozess zu kapern, um die Generation genauer zu steuern. Konkret bedeutet dies, dass der Rückwärtsprozess nicht zwangsläufig bei $t=T$ beginnen muss. Tatsächlich kann ein beliebiges $t$ als Startpunkt gewählt werden, solange $x_t$ bekannt ist. Da der Rückwärtsprozess zu jedem Zeitschritt versucht den selben Zeitschritt im Forwärtsprozess zu approximieren kann für $x_t$ einfach $q(x_t|x_0)$ genutzt werden. Einfacher ausgedrückt heißt dies, dass ein Eingabedatum entsprechend Zeitschritt $t$ verrauscht werden kann und anschließend umgehend in den Rückwärtsprozess gegeben werden kann um bei $t$ zu starten. Dies hat zur Folge, dass Samples, die auf diese Weise erzeugt wurden, dem Eingegebenen Datenpunkt ähneln. Die Stärke dieser Ähnlichkeit kann dabei über $t$ festgelegt werden - je kleiner $t$ desto weniger werden die Daten durch Rauschen zerstört, desto größer die Ähnlichkeit.


%%%%%%%%%%%%%%%%%%%%%%%%%%%%%%%%%%%%%%%%%%%%%%%%%%%%%%%%%%%%%%%%%%%%%%%%%%%%%%%
\subsubsection{Inpainting}

Ebenfalls ist es möglich die Generierung lokal zu begrenzen, sodass gezielt nur Bereiche in einem Eingabedatum durch die Generierung betroffen werden. Hierfür wird zuerst eine Maske definiert, welche die gleiche Dimension der Eingabedaten hat. Anschließend erfolgt der Rückwärtsprozess. Hierbei wird nun in jedem Schritt der maskierte Bereich von $p_\theta(x_{t-1} | x_{t})$ durch $q(x_{t-1}|x_0)$ ersetzt. Auf diese Weise bleiben die maskierten Bereiche der Eingabedaten unverändert wärend die Unmaskierten generiert wurden. \\
Aufgrund des Umstands, dass diese Ersetzung in jedem Zeitschritt stattfindet ist der Übergang zwischen Ursprungs- und neuen Daten fließend und kohärent, selbst wenn sich hierbei die Kontrollsignale unterscheiden.

%%%%%%%%%%%%%%%%%%%%%%%%%%%%%%%%%%%%%%%%%%%%%%%%%%%%%%%%%%%%%%%%%%%%%%%%%%%%%%%
\subsubsection{Diffusion Transformer}
\label{subsubsec:DiT}

\ac{DiT}\footnote{
    Peebles, Xie: Scalable Diffusion Models with Transformers
    \cite{peebles2023scalable}
} stellen eine Alternative zu U-Nets für die Umsetzung der Modellarchitektur von \ac{DM}s dar. \ac{DiT}s sind eine speziell für Diffusion angepasste Variante der sogenannten Vision Transformer\footnote{
    Dosovitskiy et al.: An Image is Worth 16x16 Words 
    \cite{dosovitskiy2021imageworth16x16words}
}.\\
Ihre grundlegende Funktionsweise ist dabei stark verwandt zu der von normalen Transformern. Die Eingabedaten werden zunächst in kleinere Patches, vergleichbar mit Tokens, unterteilt. Diese Patches werden anschließend mit einem Positional Encoding versehen, welches beispielsweise bei Bildern auch mehrdimensional sein kann. Diese angereicherten Patches werden anschließend in vielen Iterationen mittels sogenannter \ac{DiT} Blöcke, welche aus abwechselnden Self-Attention und AdaLN-Blöcken beziehungsweise Cross-Attention-Blöcken bestehen, verarbeitet. Abschließend werden die einzelnen Patches wieder zu ihrer ursprünglichen Anordnung zusammengefügt. \\
Diese Methode hat sich in einigen Kontexten als Leistungsfähiger im Vergleich zu U-Nets erwiesen. Allerdings erfordern sie ebenfalls mehr Rechenkapazität um vergleichbare oder bessere Ergebnisse zu erzeugen. Aufgrund dieser Leistungsfähigkeit werden \ac{DiT}s in einigen der aktuell leistungsfähigsten Modellen wie beispielsweise Stable Diffusion 3 verwendet.


% LDM %%%%%%%%%%%%%%%%%%%%%%%%%%%%%%%%%%%%%%%%%%%%%%%%%%%%%%%%%%%%%%%%%%%%%%%%%
\subsubsection{Latent Diffusion}

Die bisher vorgestellte Variante von DMs operiert ausschließlich in der Datendomäne. Dies ist insbesondere bei hohen Dimensionalitäten enorm Rechenintensiv, sowohl beim Training als auch bei der Inferenz. Latent Diffusion Modelle (LDM)\footnote{
    Vgl. Rombach et al.: Latent Diffusion Models
    \cite{rombach2022high}
} sind ein Ansatz um dieses Problem zu mitigieren. \\
Grundlegend bestehen LDMs aus zwei Komponenten: Einem vortrainierten VAE, sowie einem DM wie es in bereits beschrieben wurde. \\
Die Rolle des VAEs hierbei ist die Eingangsdaten in ihrer dimensionalität zu Reduzieren, sodass das DM auf signifikant weniger Dimensionen operiert. Da der nötige Rechenaufwand des DMs in aller Regel, insbesondere durch die sequenzielle Verarbeitung während der Inferenz, deutlich höher ist als die Ver- und Entschlüsselung durch einen VAE, reduziert dies den Aufwand enorm. \footnote{
    Vgl. Peebles, Xie: Diffusion Transformers, S. 8
    \cite{peebles2023scalable}
} \\
Die grundlegenden Algorithmen für Training und Inferenz bleiben hierbei weitestgehend unverändert. Beim Training müssen die Trainingsdaten nur zunächst kodiert werden und können dann genutzt werden. Bei der Inferenz muss lediglich das Resultat des Reverse-Prozesses abschließend vom \ac{VAE} entschlüsselt werden.

%%%%%%%%%%%%%%%%%%%%%%%%%%%%%%%%%%%%%%%%%%%%%%%%%%%%%%%%%%%%%%%%%%%%%%%%%%%%%%%
% PTG
%%%%%%%%%%%%%%%%%%%%%%%%%%%%%%%%%%%%%%%%%%%%%%%%%%%%%%%%%%%%%%%%%%%%%%%%%%%%%%%

\section{Terrain-Generierung}

Terrain-Generierung ist ein zentrales Gebiet der Computergraphik. Sie umfasst eine vielzahl von Algorithmen, Methoden und Datenstrukturen, welche für das Erzeugen von virtuellen Landschaften genutzt werden.

\subsection{Digital Elevation Models}

\textit{Digital Elevation Models} (DEMs) sind eine Datenstruktur für die Repräsentation von Höhenwerten einer Oberfläche. In ihnen werden beispielsweise Terrains somit über ihr Oberfläche definiert. Durch die Abbildung einer solchen dreidimensionalen Struktur auf zwei Dimensionen haben DEMs zwar den Nachteil, dass keine überhängenden Strukturen abgebildet werden können, da diese allerdings bei Landschaften nur sehr selten vorkommen, ist dies, insbesondere größeren Gebieten oft zu vernachlässigen.\\
Eine Variante von DEMs, welche die Höhenwerte den koordinaten eines zweidimensionalen Rasters zuordnet, wird im Bereich der Computergraphik auch oft als \textit{Heightmap} bezeichnet. Diese, sehr simple Datenstruktur, ist einfach und effizient zu Verarbeiten, weswegen sie oft in der Terraingenerierung angewandt wird. Aufgrund dieser Repräsentation als zweidimensionales Array sind DEMs insbesondere für die Verwendung mit Bildsynthese-Ansätzen der generativen KI geeigenet.

\subsection{Rauschbasierte Generierung}

In einigen Anwendungsgebieten, wie beispielsweise einer Terrain-Generierung zur Laufzeit eines Videospiels, ist es notwendig, dass die Algorithmen für die Erstellung der Landschaft so einfach wie möglich zu berechnen sind. Komplizierte Simulationen sind somit nicht Anwendbar. \\ 
Eine weit verbreitete alternative ist die Nutzung von sogennanten Rauschfunktionen. Solche Funktionen bilden ihre Parameter auf Zufallswerte, beziehungsweise Pseudozufallswerte ab. Angewandt auf DEMs sind diese Rauschfunktionen zweidimensional und bilden die $x$ und $y$ Koordinaten des Bildes auf einen zufälligen Höhenwert ab. \\ 
Reines statistisches Rauschen, welches beispielsweise auf einer Normalverteilung basiert, ist jedoch zu unstrukturiert um überzeugende Terrains erzeugen zu können - es fehlen größere Strukturen und Übergänge. Aus eben diesem Grund verwendet man häufig für Landschaften eine besondere Klasse von Rauschen - sogenannte \textit{Gitterrauschfunktionen}. Diese weisen Zufallswerte nun nicht mehr jedem einzelnen Bildpunkt zu. Stattdessen unterteilen sie das Bild zunächst in ein Gitter und bilden anschließend lediglich die Eckpunkte dieses Gitters zufällig ab. Nun kann zwischen diesen Werten interpoliert werden, um ein kohärenteres Zufallsbild zu erzeugen.\footnote{
    Vgl. Lagae et al.: A Survey of Procedural Noise Functions, S. 4
    \cite{https://doi.org/10.1111/j.1467-8659.2010.01827.x}
}  

\subsubsection{Perlin Rauschen}

Eine Verschärfung des Gitterrauschens ist das Gittergradientenrauschen. Hierbei werden jedem Eckpunkt des Gitters Zufallsgradienten, anstelle von Zufallswerten, zugeordnet. Die wohl populärste Methode dieser Kategorie ist das sogenannte Perlin Rauschen\footnote{
    Perlin: An Image Synthesizer
    \cite{perlin1985image}
}. \\
Zunächst werden allen Eckpunkten des Gitters pseudozufällig ausgerichtete Gradienten mit länge eins zugewiesen. Die Bestimmung eines Pixelwertes funktioniert nun wie folgt:
\begin{enumerate}
    \item Ermittle die Gradienten der Eckpunkte, die zur Gitterzelle dieses Punktes gehören.
    \item Berechne das Skalarprodukt zwischen jedem dieser Gradienten und dem Verbindungsvektor des jeweiligen Eckpunktes zur relativen Position des Pixels innerhalb der Gitterzelle.
    \item Interpoliere nun systematisch zwischen je zwei Skalarprodukten, beziehungswiese Interpolationen, $g_1$ und $g_2$, anhand der relevanten koordinate der Position $x$ des Pixels, mit der folgenden Funktion:
    \begin{equation}
        f_\text{interp}(x, g_1, g_2) = (g_2 - g_1)(6x^5 - 15x^4 + 10x^3) 
        + g_1
    \end{equation}
\end{enumerate}
Dieser Algorthmus eignet sich ebenfalls zur Generierung von unendlichen Rauschfeldern, sofern dafür gesorgt ist, dass gleichen Eckpunkten immer die gleichen pseudozufälligen Gradienten zugewiesen werden. 


\subsubsection{Fraktales Rauschen}

Eine einfache Rauschfunktion ist in ihrem Detailgrad fest an ihre Definition gebunden. Beim Perlin Rauschen beispielsweise ist er definiert über die Auflösung des Gitternetzes. Somit ist es hier nicht möglich feine und große Strukturen gleichzeitig abzubilden. Eine oft angewandte Lösung für dieses Dilemma ist die Kombination der gleichen Rauschfunktion mit unterschiedlichen Amplituden und Frequenzen, sogenanntem fraktalen Rauschen. \\
Hierbei werden hier $O$ Oktaven einer Rauschfunktion $f$ summiert. Die jeweilige Änderung der Amplitude und Frequenz in jeder Oktave werden über die Persistenz $p$ und Lakunarität $l$ definiert:  
\begin{equation}
    f_\text{fractal}(x) = \sum_{o=0}^{O-1} p^{o}f(l^ox)
\end{equation}
Übliche Belegungen sind hierbei $l=2$ und $p=\frac{1}{2}$. Die Frequenz wird also in jeder Oktave verdoppelt, wohingegen die Amplitude halbiert wird.

\subsection{KI-Basierte Terraingenerierung}

Nebst Rauschfunktionen und Simulationen existieren in der Terraingenerierung ebenfalls ansätze, welche Methoden der generativen KI anwenden, um die Beschaffenheit von Landschaften zu erlernen.

\subsubsection{Generierung mit GAN}

Vlt ischt das hier doch net so wichtig


\subsubsection{Terraingenerierung mit Diffusionsmodellen}

Diffusionsmodelle haben erst vor wenigen Jahren die generative Qualität von GANs übertreffen können. Dementsprechend ist im Bereich der Terraingenerierung nur wenig mit ihnen experimentiert worden. \\
Die erste Veröffentlichung hierzu erfolgte von Lochner et al.\footnote{
    Vgl. Lochner et al.: Interactive Terrain Authoring using Diffusion Models
    \cite{lochner2023interactive}
}. Ihr vorgeschlagenes Diffusionsmodell operiert im Bildraum und wurde auf $8\times8\text{km}$ DEMs trainiert. Als Konditionierung wurden hierbei Skizzen von Landschaftseigenschaften, auch Signaturen genannt, verwendet. Diese wurden für das Training algorithmisch aus den DEMs ermittelt. Sie erlauben bei der Generierung eine Interaktive Definition von Klippen, Gebirgskämmen, Abflüssen und flachen Gebieten. Hierbei kann bei der Generierung zwischen verschiedenen Geländetypen ausgewählt werden wie Gebirgen, Canyons und Hügellandschaften. Die Autoren erwähnen ebenfalls, dass die interaktive Kontrolle der Generation inkonsistente Ergebnisse erzeugt.\\
Hu et al.\footnote{
    Vgl. Hu et al.: Terrain Generation with Geological Sketch Guidance
    \cite{hu2024terrain}
} folgen einem ähnlichen Ansatz, wenn auch auf deutlich größeren Gebieten von $921\times921\text{km}$. Sie verwenden ebenfalls Signaturskizzen zur Steuerung des Terrains. Allerdings verwenden sie mehrere vortrainierte Autoencoder\footnote{
    Es handelt sich hierbei explizit um einen AE. Also kein VAE oder VAE-GAN.
} um die Eingabedaten zu komprimieren. Dabei ist jeder AE für die Kodierung eines einzigen Typs von Signatur, beispielsweise Flüssen, zuständig. Zusätzlich nutzen sie drei unterschiedliche DMs um unterschiedliche Detailgrade in der Terrainstruktur abbilden zu können. \\
Jain, Sharma und Rajan\footnote{
    Vgl. Jain, Sharma, Rajan: Procedural Infinite Terrain Generation with Diffusion Models
    \cite{jain2022adaptive}
} fokussieren sich hingegen auf eine unendliche Generierung. Dazu trainieren sie zunächst mehrere DMs auf $512\times512\text{m}$ DEMs welche jeweils unterschiedliche Frequenzen der DEMs Erlernen. Dies soll verschiedene sogenannte Levels of Detail (LOD) abbilden, um Echtzeitgenerierung zu ermöglichen. Dazu werden sie jeweils mittels Fourriertransformation in einzelne Frequenzbereiche aufgeteilt. Beim Sampling werden diese Frequenzbilder nach der Generation wieder zusammengefügt. Die sich hieraus ergebenen, einzelnen, Landschaftsausschnitte werden von den Autoren über Kernelblending mit Perlin Rauschen nahtlos zusammengefügt. 




%%%%%%%%%%%%%%%%%%%%%%%%%%%%%%%%%%%%%%%%%%%%%%%%%%%%%%%%%%%%%%%%%%%%%%%%%%%%%%%
% Geomorphologie
%%%%%%%%%%%%%%%%%%%%%%%%%%%%%%%%%%%%%%%%%%%%%%%%%%%%%%%%%%%%%%%%%%%%%%%%%%%%%%%

\section{Geomorphologie}

Die Geomorphologie studiert die phyischen Oberflächen-Eigenschaften von Landschaften und untersucht die ihrer Entstehung zugrundeliegenden Prozesse\footnote{
    Hugget: Fundamentals of Geomorphology, S. 3
    \cite{huggett2022fundamentals}
}. Als integrativer Wissenschaftszweig verbindet sie unter anderem Aspekte der Geologie, Klimatologie und Hydrologie, um die Beschaffenheit von Gebirgsketten, Tälern und allem dazwischen zu analysieren. Diese Prozesse umfassen eine Vielzahl natürlicher Mechanismen, darunter beispielsweise Erosion, Sedimentation und chemische Verwitterung. \\
In der vorliegenden Arbeit spielen zwei Bereiche der Geomorphologie eine besondere Rolle. Zum einen die topographische Klassifikation von Landmassen, welche eine systematische Beschreibung verschiedener Landschaftsformen ermöglicht. Zum anderen wird der Einfluss klimatischer Bedingungen betrachtet, da Klimafaktoren wie Temperatur und Niederschlag maßgeblich zur Entwicklung geomorphologischer Strukturen beitragen\footnote{
    Hugget: Fundamentals of Geomorphology, S. 17
    \cite{huggett2022fundamentals}
}.


\subsection{Terrain Klassifizierung}

\subsection{Klima Klassifizierung}

