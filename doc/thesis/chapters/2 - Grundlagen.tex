\chapter{Grundlagen}

In diesem Kapitel werden die Grundlagen der verwendeten Themen und Technologien --- Begriffe und Sachverhalte --- erläutert, die für das Verständnis der folgenden Arbeit notwendig sind.
Die Grundlagen werden ohne Bezug auf das konkrete Projekt, allgemein dargestellt. Sie sind zielgerichtet auf das Verständnis des Hauptteils bezogen.

% \paragraphnl ist ein custom command. Ein \paragraph plus ein line-break / new-line.
% (Nach \paragraphnl{Paragraph} muss eine Zeile frei gelassen werden.)


%%%%%%%%%%%%%%%%%%%%%%%%%%%%%%%%%%%%%%%%%%%%%%%%%%%%%%%%%%%%%%%%%%%%%%%%%%%%%%%
% Mathematics
%%%%%%%%%%%%%%%%%%%%%%%%%%%%%%%%%%%%%%%%%%%%%%%%%%%%%%%%%%%%%%%%%%%%%%%%%%%%%%%
\section{Mathematik}

\subsection{Wahrscheinlichkeits-Dichte}

\subsection{Score}

\subsection{Likelihood}

\subsubsection{Likelihood-Estimation}

\subsection{Kullback-Leibler-Divergenz}


%%%%%%%%%%%%%%%%%%%%%%%%%%%%%%%%%%%%%%%%%%%%%%%%%%%%%%%%%%%%%%%%%%%%%%%%%%%%%%%
% AI
%%%%%%%%%%%%%%%%%%%%%%%%%%%%%%%%%%%%%%%%%%%%%%%%%%%%%%%%%%%%%%%%%%%%%%%%%%%%%%%

\section{Generative Künstliche Intelligenz}

$$
2-2 = 1
$$

% Autoencoder %%%%%%%%%%%%%%%%%%%%%%%%%%%%%%%%%%%%%%%%%%%%%%%%%%%%%%%%%%%%%%%%%
\subsection{Autoencoder}

% GAN %%%%%%%%%%%%%%%%%%%%%%%%%%%%%%%%%%%%%%%%%%%%%%%%%%%%%%%%%%%%%%%%%%%%%%%%%
\subsection{Generative Adversarial Networks}

% VAE %%%%%%%%%%%%%%%%%%%%%%%%%%%%%%%%%%%%%%%%%%%%%%%%%%%%%%%%%%%%%%%%%%%%%%%%%
\subsection{Variatonal Autoencoder}

\subsubsection{Evidence Lower Bound}

\subsubsection{VAE-GAN}

\subsubsection{Visual Similarity Metrics}

% DDPM %%%%%%%%%%%%%%%%%%%%%%%%%%%%%%%%%%%%%%%%%%%%%%%%%%%%%%%%%%%%%%%%%%%%%%%%
\subsection{Denoising Diffusion Probabilistic Models}

\subsubsection{Inpainting}

\subsubsection{Improved DDPMs}

\subsubsection{Classifier Free Guidance}

\subsubsection{Adaptive Group Normalization}

\subsubsection{Inpainting}

\subsubsection{UNET}

\subsubsection{ResNet}

\subsubsection{Transformer}

\subsubsection{Attention}

\subsubsection{Diffusion Transformer}


% LDM %%%%%%%%%%%%%%%%%%%%%%%%%%%%%%%%%%%%%%%%%%%%%%%%%%%%%%%%%%%%%%%%%%%%%%%%%
\subsection{Latent Diffusion Models}


%%%%%%%%%%%%%%%%%%%%%%%%%%%%%%%%%%%%%%%%%%%%%%%%%%%%%%%%%%%%%%%%%%%%%%%%%%%%%%%
% PTG
%%%%%%%%%%%%%%%%%%%%%%%%%%%%%%%%%%%%%%%%%%%%%%%%%%%%%%%%%%%%%%%%%%%%%%%%%%%%%%%

\section{Prozedurale Terrain Generierung}

Aliquam molestie fermentum vestibulum. Cras egestas molestie ipsum, vitae malesuada ante consectetur id. In turpis neque, pharetra eget neque vel, rhoncus tincidunt ex. 

\subsection{Digital Elevation Models}

\subsection{Rauschbasierte Generierung}


\subsubsection{Perlin Rauschen}

\subsubsection{Fraktales Perlin Rauschen}

Sed lacinia fermentum odio quis faucibus. Phasellus blandit orci vitae ipsum rutrum aliquam. 

\subsection{KI-Basierte Generierung}

Fusce ipsum nisl, luctus in interdum non, sodales sed lacus. 

\subsubsection{Generierung mit GAN}

Curabitur tincidunt mauris ac venenatis accumsan. 

\subsubsection{Generierung mit Diffusion}

Mauris efficitur sit amet mauris in sodales. 

\subsubsection{Unter-Unterabschnitt\#3}

%%%%%%%%%%%%%%%%%%%%%%%%%%%%%%%%%%%%%%%%%%%%%%%%%%%%%%%%%%%%%%%%%%%%%%%%%%%%%%%
% Geomorphologie
%%%%%%%%%%%%%%%%%%%%%%%%%%%%%%%%%%%%%%%%%%%%%%%%%%%%%%%%%%%%%%%%%%%%%%%%%%%%%%%

\section{Geomorphologie}


\subsection{Terrain Klassifizierung}

\subsection{Klima Klassifizierung}

