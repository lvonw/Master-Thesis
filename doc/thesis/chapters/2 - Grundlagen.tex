\chapter{Grundlagen}

In diesem Kapitel werden die Grundlagen der verwendeten Themen und Technologien --- Begriffe und Sachverhalte --- erläutert, die für das Verständnis der folgenden Arbeit notwendig sind.
Die Grundlagen werden ohne Bezug auf das konkrete Projekt, allgemein dargestellt. Sie sind zielgerichtet auf das Verständnis des Hauptteils bezogen.

% \paragraphnl ist ein custom command. Ein \paragraph plus ein line-break / new-line.
% (Nach \paragraphnl{Paragraph} muss eine Zeile frei gelassen werden.)


%%%%%%%%%%%%%%%%%%%%%%%%%%%%%%%%%%%%%%%%%%%%%%%%%%%%%%%%%%%%%%%%%%%%%%%%%%%%%%%
% Mathematics
%%%%%%%%%%%%%%%%%%%%%%%%%%%%%%%%%%%%%%%%%%%%%%%%%%%%%%%%%%%%%%%%%%%%%%%%%%%%%%%
\section{Mathematik}

\subsection{Wahrscheinlichkeits-Dichte}

\subsection{Score}

\subsection{Likelihood}

\subsubsection{Likelihood-Estimation}

\subsection{Kullback-Leibler-Divergenz}

Seien $P(x)$ und $Q(x)$ zwei Wahrscheinlichkeitsverteilungen, so ist die KL-Divergenz definiert als \footnote{
    Vgl. Goodfellow, Bengio, Courville: Deep Learning, S. 74 f.
    \cite{Goodfellow-et-al-2016}
}:

\begin{equation}
D_\text{KL}(P||Q) 
:= \mathbb{E}_{x \sim P} \left [
\log \frac{P(x)} {Q(x)}
\right ]
= \mathbb{E}_{x \sim P} \left [ 
    \log P(x) - \log Q(x)
    \right ]
\end{equation}

\subsection{Erwartungswert}

Der Erwartungswert einer Zufallsvariable $X$ unter einer Wahrscheinlichkeitsverteilung $p(X)$ ist gegeben als: 
\begin{equation}
    \mathbb E(X) := \int x p(x) dx
\end{equation}
In vielen Publikationen im Bereich der künstlichen Intelligenz wird der Erwartungswert auch wie folgt angegeben:
\begin{equation}
    \mathbb E_{x \sim p}[x] := \int x p(x) dx
\end{equation}
Diese Definition kann ebenfalls um den Erwartungswert einer Funktion $f(x)$ angewandt werden, wenn x aus der Wahrscheinlichkeitsverteilung $p(X)$ entnommen wird. 
\begin{equation}
    \mathbb E_{x \sim p}[f(x)] := \int f(x) p(x) dx
\end{equation}


%%%%%%%%%%%%%%%%%%%%%%%%%%%%%%%%%%%%%%%%%%%%%%%%%%%%%%%%%%%%%%%%%%%%%%%%%%%%%%%
% AI
%%%%%%%%%%%%%%%%%%%%%%%%%%%%%%%%%%%%%%%%%%%%%%%%%%%%%%%%%%%%%%%%%%%%%%%%%%%%%%%

\section{Generative Künstliche Intelligenz}

Modelle der generativen KI versuchen im Allgemeinen die Wahrscheinlichkeitsverteilung der Eingabedaten, oft auch als Datenverteilung bezeichnet, zu modellieren. Sie unterscheiden sich somit von diskreminierenden Ansätzen, welche lediglich Eigenschaften dieser Verteilung erlernen. \\
Es kann schwierig fallen, intuitiv die Beziehung zwischen Datenverteilung und Modell nachzuvollziehen. Um diesen Sachverhalt zu verdeutlichen, folgt ein einfaches Beispiel:

Ein Münzwurf soll modelliert werden. Das Ergebnis von Münzwürfen ist mit der Wahrscheinlichkeitsverteilung $p(X)$ beschrieben wobei $x \in [\text{Kopf}, \text{Zahl}]$. Um zu verdeutlichen, dass es sich bei $p(X)$ um die Datenverteilung handelt wird sie oft auch als $p_\text{data}(X)$ bezeichnet. \\
Das Modell, welches den Münzwurf modellieren soll wird $p_\theta(X)$ bezeichnet, wobei $\theta$ die Parameter dieses Modells darstellt. Das Ziel ist, dass Münzwürfe des Modells, nicht von tatsächlichen Münzwürfen unterschieden werden können. Einfacher ausgedrückt, müssen die Wahrscheinlichkeiten für Kopf und Zahl jeweils gleich sein: $p_\theta(X) \overset{!}{=} p_\text{data}(X)$. \\ 
Um dies zielgerichtet erfüllen zu können benötigen wir eine Metrik, welche angibt, wie gut das Modell die Datenverteilung approximiert. Hierfür kann beispielsweise die Distanz zwischen beiden Verteilungen genommen werden. Eine Möglichkeit diese zu Berechnen ist die KL-Divergenz $D_\text{KL}(p_\text{Data}(X)||p_\theta(X))$. Um eine solche Berechnung durchzuführen muss $p_\text{Data}(X)$ allerdings bekannt sein. Grundsätzlich ist dies allerdings für Datenverteilungen nicht möglich. \\
Die Minimierung der KL-Divergenz ist allerdings mathematisch äquivalent zur minimierung Kreuzentropie\footnote{
    Goodfellow, Bengio, Courville: Deep Learning, S. 75
    \cite{Goodfellow-et-al-2016}
}. 
Dies entspricht wiederum der maximierung der Log-Likelihood und ist somit als Trainingskriterium bei Modellen bekannt.
Damit $p_\theta(X)$ $p_\text{data}(X)$ erlernen kann, werden zunächst Beispiele benötigt. Dazu wird mehrfach eine Münze geworfen. Die Menge all dieser Ergebnisse sind die Trainingsdaten. Ein einzelner Münzwurf $x_n$ der Menge entspricht einem Sample, welches der Datenverteilung entnommen wurde $x_n \sim p_\text{data}(X)$. Nun kann über die Maximierung der Log-Likelihood normales Training vollzogen werden.

Dieses Beispiel kann auf andere Anwendungsfälle übertragen werden. In der Bildysnthese entspricht die Datenverteilung beispielsweise einer Verteilung von Farbwerten auf $n \times m$ Pixeln, im Gegensatz zu einem Münzwurf auf Kopf oder Zahl.

% Autoencoder %%%%%%%%%%%%%%%%%%%%%%%%%%%%%%%%%%%%%%%%%%%%%%%%%%%%%%%%%%%%%%%%%
\subsection{Autoencoder}

Autoencoder (AE) sind streng genommen keine generativen Modellen, da sie lediglich rekonstruieren und keine neuen Daten erzeugen. Sie stellen allerdings die Grundlage für die Variational Autoencoder dar, welche generativ sind und im folgenden Unterabschnitt behandelt werden. \\ 
AEs haben als grundlegendes Ziel haben eine Eingabe in der dimensionalität zu reduzieren und anschließend wieder zur ursprünglichen Eingabe zu rekonstruieren. Man kann sie sich entsprechend als eine erlernte Variante von Kompressionsalgorithmen wie ZIP vorstellen. 
Zu diesem Zweck bestehen AE aus zwei Komponenten\footnote{
    Die Encoder und Decoder Komponenten werden hier als $q$ und $p$ bezeichnet um die Verbindung zu Variational Autoencodern zu verdeutlichen. In vielen Publikationen werden sie allerdings $E$ und $D$ denotiert.    
}:
\begin{itemize}
    \item \textbf{Encoder}: \\
    Erhält als Eingabe Samples der Datenverteilung und komprimiert sie zu einem Code $h$. Sie werden folgend als $q_\phi(x)$ bezeichnet
    \item \textbf{Decoder}: \\
    Erhält als Eingabe den versteckten Zustand und rekonstruiert das ursprüngliche Sample der Datenverteilung möglichst verlustfrei. Diese Komponente modelliert die Datenverteilung auf basis des Codes und wird entsprechend folgend $p_\theta(h)$ denotiert.
\end{itemize}

Das Training eines solchen AEs ist sehr simpel. Man minimiere lediglich die Distanz zwischen Eingabedaten und Rekonstruktionen. Somit ergibt sich folgendes Optimierungsziel:

\begin{equation}
    L_{\phi, \theta}(x) := \mathbb E _{x \sim p_{data}} 
    \left [ 
        d(x, p_\theta(q_\phi(x)))
    \right ]
\end{equation}

% VAE %%%%%%%%%%%%%%%%%%%%%%%%%%%%%%%%%%%%%%%%%%%%%%%%%%%%%%%%%%%%%%%%%%%%%%%%%
\subsubsection{Variatonal Autoencoder}

Eine spezialisierte Variante der Autoencoder sind sogenannte Variational Autoencoder (VAE)\footnote{
    Kingma, Welling: Auto-Encoding Variational Bayes
    \cite{kingma2013auto}
}. Ihre Erweiterung der Autoencoder besteht in der Einführung einer Bedingung an den Code. Dieser kann in im ursprünglichen AE ein praktisch beliebiger Vektor sein, solange er bei der Entschlüsselung von Vorteil ist. Je nach Anwendungsfall kann es allerdings hilfreich sein, wenn der Code selbst auch eine semantisch Aussagekräftige Darstellung der Eingabe ist. Dies kann den Code, je nach Anwendungsfall, zu einem geringerdimensionalen Substitut für die Ursprungsdaten machen. Hierbei ist allerdings wichtig zu beachten, dass es sich hierbei in aller Regel nicht um menschenverständliche Eigenschaften handelt. \\
Zu diesem Zweck wird der latente Ergebnisraum des Encoders mit einer beliebigen Wahrscheinlichkeitsverteilung $p(z)$ kombiniert. Eingabedaten werden somit also einen latenten Raum $Z$ abgebildet, welcher grob $p(z)$ entspricht. \\
Die zugrundeliegende Aussage die somit getroffen wird ist, dass jede Komponente des Codes nicht nur eine Kompression der Eingabe ist, sondern auch selbst einer Wahrscheinlichkeitsverteilung folgt. Hierbei wird angenommen, dass Merkmale der Datenverteilung, die beispielsweise einer Normalverteilung folgen, eine strukturierte und nützliche Darstellung der Eingabedaten ermöglichen. Die so entstehende Verteilung hat die folgende Form:
\begin{equation}
    p_\theta(x) = \int_z p_\theta(x, z) dz
\end{equation}
Diese Anpassung hat zur Folge, dass das Training eines VAE ein neues Optimierungsziel benötigt. Der erste Gedanke ist, die Log-Likelihood der Verteilung $p_\theta(X)$ zu maximieren. Dies ist aufgrund des Integrals über den latenten Raum nicht effizient möglich. Dieses Problem wird gelöst, indem eine zweite Quantität maximiert wird, welche garantiert kleiner als die Log-Likelihood ist. Diese Quantität ist das Optimierungsziel und wird allgemein die \textit{Evidence Lower Bound} (ELBO) genannt. Sie ist gegeben als:
\begin{equation}
    \mathcal L_{\phi, \theta}(x) = \mathbb E_{q_{\phi}(z)}
    \left [
        \log p_\theta(x|z) - D_\text{KL} (\log q_{\phi}(z|x) || p(z))
    \right ]
\end{equation}
$p(z)$ ist hierbei die, für die Verteilung der Merkmale angenommene Verteilung. In der Regel handelt es sich hierbei um eine Standardnormalverteilung $\mathcal N(0; I)$. Die Herleitung der ELBO ausgehend von der Log-Likelihood ist sehr komplex und dem Verständnis nicht weiter zutragend. Eine Betrachtung der einzelnen Therme bietet allerdings Aufschluss über die Bedeutung der ELBO:
\begin{itemize}
    \item $\log p_\theta(x|z)$: \\
    Hierbei handelt es sich um die Log-Likelihood der Eingabedaten unter der Bedingung des jeweiligen latenten Codes. Einfacher ausgedrückt stellt dies die Übereinstimmung der Eingabe und der Rekonstruktion sicher. Somit ist dies lediglich das Optimierungsziel des ursprünglichen AE.
    \item $D_\text{KL} (\log q_{\phi}(z|x) || p(z))$: \\
    Diese KL-Divergenz stellt eine Regularisierungstherm dar. Sie fordert, dass die Verteilung $q_{\phi}(z|x)$, sprich der latente Ergebnisraum des Encoders, der angenommenen zugrundeliegenden Verteilung $p(z)$ entspricht.
\end{itemize}
Tatsächlich wurde also, gegenüber des ursprünglichen Optimierungsziels nur ein Regularisierungstherm hinzugefügt.



% GAN %%%%%%%%%%%%%%%%%%%%%%%%%%%%%%%%%%%%%%%%%%%%%%%%%%%%%%%%%%%%%%%%%%%%%%%%%
\subsection{Generative Adversarial Networks}


\subsubsection{VAE-GAN}

\subsubsection{Visual Similarity Metrics}


% SUPPORTING MODELS %%%%%%%%%%%%%%%%%%%%%%%%%%%%%%%%%%%%%%%%%%%%%%%%%%%%%%%%%%%

\subsection{UNET}

\subsection{Transformer}

\subsubsection{Attention}

% DDPM %%%%%%%%%%%%%%%%%%%%%%%%%%%%%%%%%%%%%%%%%%%%%%%%%%%%%%%%%%%%%%%%%%%%%%%%
\subsection{Denoising Diffusion Probabilistic Models}

\subsubsection{Improved DDPMs}

\subsubsection{Classifier Free Guidance}

\subsubsection{Adaptive Group Normalization}

\subsubsection{Inpainting}


\subsubsection{Diffusion Transformer}


% LDM %%%%%%%%%%%%%%%%%%%%%%%%%%%%%%%%%%%%%%%%%%%%%%%%%%%%%%%%%%%%%%%%%%%%%%%%%
\subsection{Latent Diffusion Models}

Die bisher vorgestellte Variante von Diffusions-Modellen operiert ausschließlich in der Bilddomäne. Dies ist insbesonder bei hohen Auflösungen enorm Rechenintensiv, sowohl beim Training als auch bei der Inferenz. Latent Diffusion Modelle (LDM)\footnote{
    Vgl. Rombach et al.: Latent Diffusion Models
    \cite{rombach2022high}
}  sind ein Ansatz um dieses Problem zu mitigieren. \\
Grundlegend bestehen LDMs aus zwei Komponenten: Ein, auf den für das DM gedachten Trainingsdaten, vortrainierter VAE, sowie ein DM, wie es in bereits 2.2.3 beschrieben wurde. \\
Die Rolle des VAEs hierbei ist die Eingangsdaten in ihrer dimensionalität zu Reduzieren, sodass das DM auf signifikant weniger Dimensionen operiert. Da der nötige Rechenaufwand des DMs in aller Regel, insbesondere durch die sequenzielle Verarbeitung während der Inferenz, deutlich höher ist als die Ver- und Entschlüsselung durch einen VAE reduziert dies den Aufwand enorm.  \footnote{
    Vgl. Peebles, Xie: Diffusion Transformers, S. 8
    \cite{peebles2023scalable}
} \\
Die grundlegenden Algorithmen für Training und Inferenz bleiben hierbei weitestgehend unverändert. Beim Training müssen die Trainingsdaten nur zunächst kodiert werden und können dann genutzt werden. Bei der Inferenz muss lediglich das Resultat des Reverse-Prozesses einmal entschlüsselt werden.

%%%%%%%%%%%%%%%%%%%%%%%%%%%%%%%%%%%%%%%%%%%%%%%%%%%%%%%%%%%%%%%%%%%%%%%%%%%%%%%
% PTG
%%%%%%%%%%%%%%%%%%%%%%%%%%%%%%%%%%%%%%%%%%%%%%%%%%%%%%%%%%%%%%%%%%%%%%%%%%%%%%%

\section{Terrain-Generierung}

Terrain-Generierung ist ein zentrales Gebiet der Computergraphik. 

\subsection{Digital Elevation Models}

\textit{Digital Elevation Models} (DEMs) sind eine Datenstruktur für das Repräsentation von Höhenwerten. In ihnen werden beispielsweise Terrains somit über ihr Oberfläche definiert. Durch die Abbildung einer solchen dreidimensionalen Struktur auf zwei Dimensionen haben DEMs zwar den Nachteil, dass keine überhängenden Strukturen abgebildet werden können, da diese allerdings bei Landschaften nur sehr selten vorkommen, ist dies, insbesondere größeren Gebieten oft zu vernachlässigen.\\
Eine Variante von DEMs, welche die Höhenwerte den koordinaten eines zweidimensionalen Rasters zuordnet, wird im Bereich der Computergraphik auch oft als \textit{Heightmap} bezeichnet. Diese, sehr simple Datenstruktur, ist einfach und effizient zu Verarbeiten, weswegen sie oft in der Terraingenerierung angewandt wird. Aufgrund dieser Repräsentation als zweidimensionales Array sind DEMs insbesondere für die Verwendung mit Bildsynthese-Ansätzen der generativen KI geeigenet.

\subsection{Rauschbasierte Generierung}

In einigen Anwendungsgebieten, wie beispielsweise einer Terrain-Generierung zur Laufzeit eines Videospiels, ist es notwendig, dass die Algorithmen für die Erstellung der Landschaft so einfach wie möglich zu berechnen sind. Komplizierte Simulationen sind somit nicht Anwendbar. \\ 
Eine weit verbreitete alternative ist die Nutzung von sogennanten Rauschfunktionen. Solche Funktionen bilden ihre Parameter auf Zufallswerte, beziehungsweise Pseudozufallswerte ab. Angewandt auf DEMs sind diese Rauschfunktionen zweidimensional und bilden die $x$ und $y$ Koordinaten des Bildes auf einen zufälligen Höhenwert ab. \\ 
Reines statistisches Rauschen, welches beispielsweise auf einer Normalverteilung basiert, ist jedoch zu unstrukturiert um überzeugende Terrains erzeugen zu können - es fehlen größere Strukturen und Übergänge. Aus eben diesem Grund verwendet man häufig für Landschaften eine besondere Klasse von Rauschen - sogenannte \textit{Gitterrauschfunktionen}. Diese weisen Zufallswerte nun nicht mehr jedem einzelnen Bildpunkt zu. Stattdessen unterteilen sie das Bild zunächst in ein Gitter und bilden anschließend lediglich die Eckpunkte dieses Gitters zufällig ab. Nun kann zwischen diesen Werten interpoliert werden, um ein kohärenteres Zufallsbild zu erzeugen.\footnote{
    Vgl. Lagae et al.: A Survey of Procedural Noise Functions, S. 4
    \cite{https://doi.org/10.1111/j.1467-8659.2010.01827.x}
}  

\subsubsection{Perlin Rauschen}

Eine Verschärfung des Gitterrauschens ist das Gittergradientenrauschen. Hierbei werden jedem Eckpunkt des Gitters Zufallsgradienten, anstelle von Zufallswerten, zugeordnet. Die wohl populärste Methode dieser Kategorie ist das sogenannte Perlin Rauschen\footnote{
    Perlin: An Image Synthesizer
    \cite{perlin1985image}
}. 

\subsubsection{Fraktales Perlin Rauschen}

Sed lacinia fermentum odio quis faucibus. Phasellus blandit orci vitae ipsum rutrum aliquam. 

\subsection{KI-Basierte Generierung}

Fusce ipsum nisl, luctus in interdum non, sodales sed lacus. 

\subsubsection{Generierung mit GAN}

Curabitur tincidunt mauris ac venenatis accumsan. 

\subsubsection{Generierung mit Diffusion}

Mauris efficitur sit amet mauris in sodales. 

\subsubsection{Unter-Unterabschnitt\#3}

%%%%%%%%%%%%%%%%%%%%%%%%%%%%%%%%%%%%%%%%%%%%%%%%%%%%%%%%%%%%%%%%%%%%%%%%%%%%%%%
% Geomorphologie
%%%%%%%%%%%%%%%%%%%%%%%%%%%%%%%%%%%%%%%%%%%%%%%%%%%%%%%%%%%%%%%%%%%%%%%%%%%%%%%

\section{Geomorphologie}


\subsection{Terrain Klassifizierung}

\subsection{Klima Klassifizierung}

