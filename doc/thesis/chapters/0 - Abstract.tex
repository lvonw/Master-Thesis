
\begin{abstract}
\paragraphnl{Abstract}

{\small
Die Generierung von Landschaften mithilfe von Diffusionsmodellen ist ein neues und bisher wenig erforschtes Gebiet. Diese Arbeit untersucht die Eignung von Diffusionsmodellen für verschiedene Anwendungsbereiche der Terraingenerierung, basierend auf modernsten Methoden. 

Hierfür wurde zunächst ein latentes Diffusionsmodel (LDM) entwickelt, welches auf einem leistungsfähigen VAE-GAN basiert. Zwei gängige Modellarchitekturen, Diffusion Transformer (DiT) und U-Net, wurden verglichen, wobei sich zeigte, dass DiT in dieser Anwendung aufgrund signifikanter Artefakte ungeeignet ist. \\
Aufbauend auf diesem LDM wurden drei spezifische Anwendungsbereiche untersucht: unkontrollierte-, skizzenbasierte- und unendliche Generierung. Die unkontrollierte Generierung wies Qualitätsmängel auf, die auf vereinfachte und somit zu schwache Kontrollsignale zurückgeführt wurden. \\
Im Gegensatz dazu ermöglichte der Ansatz zur skizzenbasierten Generierung eine präzise Kontrolle der Höhenwerte durch die direkte Nutzung von digitalen Höhenmodellen (DEMs) anstelle von Landschaftssignaturen. Der Ansatz erlaubt sowohl grobe als auch detaillierte Skizzen und unterstützt zudem Kontrollsignale, welche sich nicht durch Signaturen abbilden lassen, wie Perlin-Rauschen. \\
Für die unendliche Generierung wurde ein neuartiger Ansatz entwickelt, der auf gitterbasierter Inpainting-Technik beruht. Hierbei werden kohärente Landschaftsabschnitte unter der Nutzung von maskierten Randbereichen generiert.

Die erzielten Ergebnisse demonstrieren die Vielseitigkeit und Flexibilität von Diffusionsmodellen für die Terraingenerierung, und bietet Perspektiven für zukünftige Entwicklungen.
}
\end{abstract}