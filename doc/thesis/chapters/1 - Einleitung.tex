\chapter{Einleitung}


%%%%%%%%%%%%%%%%%%%%%%%%%%%%%%%%%%%%%%%%%%%%%%%%%%%%%%%%%%%%%%%%%%%%%%%%%%%%%%%
% Kapitel Header  
%%%%%%%%%%%%%%%%%%%%%%%%%%%%%%%%%%%%%%%%%%%%%%%%%%%%%%%%%%%%%%%%%%%%%%%%%%%%%%%

Die Terrain-Generierung ist ein zentrales Gebiet in der Computergraphik. Sie enthält Methoden und Algorithmen, um Landschaften und ihre natürliche Strukturen wie Berge, Täler und Flüsse zu erzeugen. Sie findet überall dort Verwendung, wo virtuelle Welten entstehen, von Videospielen über Filme bis hin zu Anwendungen in der Architektur. \\
% Dabei muss häufig zwischen realistischer Erscheinung, Aufwand in der Implementierung und Laufzeiteffizienz abgewogen werden. Simple prozedurale Methoden wie fraktales Perlin- oder Voronoi-Rauschen sind zwar schnell implementiert und können in Sekundenbruchteilen riesige Gebiete erzeugen. Allerdings gehen sie hierbei erhebliche Kompromisse in Realismus und Vielfalt ein. Die Simulation von physikalischen Prozessen wie Plattentektonik oder Erosion hingegen ist aufwändig, komplex und erfordert viel Expertenwissen. \\
Generative künstliche Intelligenz, welche in den letzten Jahren enorme Fortschritte verzeichnen konnte, ist in diesem Feld eine vielversprechende und bisher nur wenig erforschte Alternative. Gerade im Bereich der Bildsynthese können sogenannte \textit{latente Diffusionsmodelle} wie Stable-Diffusion, Imagen, MidJourney und viele weitere beeindruckende Ergebnisse erzeugen, welche kaum noch von echten Bildern zu unterscheiden sind.
Ihre inhärente Fähigkeit, komplexeste Strukturen in räumlichen Daten zu erlernen und anschließend zu imitieren, bietet sich an, um die komplizierten Merkmale von Landschaften abzubilden. Dies könnte einen guten Kompromiss zwischen augenscheinlichem Realismus und verbundenem Aufwand darstellen. \\
Diese Arbeit stellt einen neuartigen Einsatz dieser Technologie und ihrer Vorteile für die Terrain-Generierung vor. Insbesondere wird hierbei der hohe Grad an Kontrolle, den Diffusionsmodelle ermöglichen, mit einer nahtlosen, zellbasierten Generierung verbunden. Dies ist eine von vergleichbaren Umsetzungen bisher unerreichte Spannweite der Einsatzmöglichkeiten im Bereich der Terraingenerierung. Für die Umsetzung kommen hierfür Methoden zum Einsatz, die auf dem neuesten Stand der Technik basieren.
 
Dieses einleitende Kapitel geht auf die Motivation ein, bietet eine genaue Problemdefinition und präzisiert die für dessen Lösung zu erreichenden Kernziele. Abschliessend wird die weitere Struktur der Arbeit dargelegt. 


%%%%%%%%%%%%%%%%%%%%%%%%%%%%%%%%%%%%%%%%%%%%%%%%%%%%%%%%%%%%%%%%%%%%%%%%%%%%%%%
% Motivation
%%%%%%%%%%%%%%%%%%%%%%%%%%%%%%%%%%%%%%%%%%%%%%%%%%%%%%%%%%%%%%%%%%%%%%%%%%%%%%%
\section{Motivation}

Landschaften üben seit jeher eine besondere Faszination auf die Menschheit aus. In jeder Kultur und Kunstform wird immer wieder versucht, die Einzigartigkeit und Schönheit von Landschaften einzufangen. Hierbei sind auch die virtuellen Welten keine Ausnahme. Filme wie David Camerons \textit{Avatar} oder Videospiele wie \textit{Minecraft} sind exemplarisch hierfür. \\
% Bisweilen muss hierbei sorgfältig zwischen realistischer Erscheinung, Aufwand in der Implementierung und Laufzeiteffizienz abgewogen werden. Denn ebenso vielfältig und komplex wie reale Terrains sind ebenfalls die zugrundeliegenden geomorphologischen Prozesse, die ihr Erscheinungsbild bestimmen. Dies macht die Generierung von realistischen Landschaften anhand akkurater Simulationen zu einer höchst anspruchsvollen und rechenintensiven Aufgabe. \\
% Zur Bewältigung dieses Auwfands werden häufig vereinfachende Methoden wie fraktales Rauschen verwendet. Diese ermöglichen zwar eine schnelle Generierung und sind in aller Regel einfach umzusetzen, bedeuten allerdings erhebliche Einbußen in Realismus und Vielfalt und sind nur sehr schwierig zu kontrollieren. \\
% Generative \ac{KI} verspricht realistische Terrains ohne Simulation zu ermöglichen. Viele der dazu untersuchten Ansätze basieren allerdings auf \textit{Generative Adverserial Networks} welche als schwierig zu trainieren und zu kontrollieren gelten. \\
% Diffusionsmodelle, eine vergleichsweise junge Klasse von generativen Deep-Learning-Modellen welche sich besonders für die Modellierung von räumlichen Daten eignet, stellen eine vielversprechende Alternative dar. \\
Zahlreiche Methoden für die Erstellung solcher digitalen Terrains wurden über Jahrzehnte hinweg bereits entwickelt. Viele ihrer Vorteile könnten durch geeignete Nutzung von Diffusionsmodellen vereint werden. Durch ihre Fähigkeit, kontrollierbar komplexeste räumliche Strukturen zu modellieren, könnten somit neue Maßstäbe in der digitalen Terraingenerierung gesetzt werden.\\
% Insbesondere können schnell und intuitiv erstellte Skizzen für die Steuerung der Synthese genutzt werden. Ebenfalls ist es möglich, unterschiedliche Landschaftspatches nahtlos ineinander übergehen zu lassen. \\
Dieses Potenzial wird in dieser Arbeit eingehend beleuchtet. Die erarbeiteten Ergebnisse sollen dabei sowohl die vielfältigen Einsatzmöglichkeiten von Diffusionsmodellen verdeutlichen als auch neue Richtungen für die zukünftige Entwicklung dieses Forschungsfeldes aufweisen.


%%%%%%%%%%%%%%%%%%%%%%%%%%%%%%%%%%%%%%%%%%%%%%%%%%%%%%%%%%%%%%%%%%%%%%%%%%%%%%%
% Problemdefinition
%%%%%%%%%%%%%%%%%%%%%%%%%%%%%%%%%%%%%%%%%%%%%%%%%%%%%%%%%%%%%%%%%%%%%%%%%%%%%%%
\section{Problemdefinition}

In der Terraingenerierung muss die Wahl der zu nutzenden Methode sorgfältig abgewogen werden. Faktoren wie Aufwand in der Implementierung, Laufzeiteffizienz, Realismus, größe der Generation oder Vielfalt sind hierbei die betreffenden Entscheidungskriterien, welche je nach Anwendungsfall gewichtet werden müssen. \\
So werden beispielsweise in Videospielen häufig rauschbasierte Ansätze gewählt. Diese sind in der Regel einfach zu implementieren, sehr schnell zu berechnen und eignen sich für praktisch unendlich große Landschaften. Diese Simplizität erfordert allerdings Kompromisse bei der Vielfalt, der Kontrollierbarkeit und dem Realismus. \\
Der Gegenpol hierzu ist eine akkurate Simulation der physikalischen Prozesse, die die Topographie eines Terrains bestimmen. Hierdurch lassen sich qualitativ hochwertige Terrains generieren, welche sich ebenfalls durch verschiedene Paremeter und zugrundeliegende Skizzen genau steuern lassen. Dies erfodert allerdings die Modellierung von enorm komplexen Systemen und benötigt somit viel Aufwand bei der Implementierung und tiefes Expertenwissen gegebenenfalls sogar bei der Nutzung. \\
In der Forschung hierzu ist es häufig das Ziel einen guten Kompromiss zwischen diesen beiden Extremen zu finden, der auf möglichst viele Bereiche anwendbar ist. Eine hierfür mögliche Option liegt im Bereich der generativen künstlichen Intelligenz. In der Vergangenheit basieren viele der dazu untersuchten Ansätze auf \textit{Generative Adverserial Networks}. Diese haben sich jedoch als schwierig zu trainieren und zu kontrollieren erwiesen. \\
Diffusionsmodelle sind eine junge Klasse von generativen Deep-Learning-Modellen. Sie eignen sich besonders für die Modellierung von räumlichen Daten und stellen eine vielversprechende Alternative dar. Sie übertreffen GANs in vielerlei Metriken wie Qualität und Vielfalt der Ergebnisse, Steuerbarkeit der Generierung und der Einfachheit des Trainings. \\
Erste Ansätze in der Terraingenerierung, welche diese Technologie nutzen, existieren bereits. Das mögliche Potenzial in Hinblick auf Vereinbarung von einfacher Kontrolle, unendlicher Landschaften sowie Realismus ist allerdings noch weitgehend unerforscht.

%%%%%%%%%%%%%%%%%%%%%%%%%%%%%%%%%%%%%%%%%%%%%%%%%%%%%%%%%%%%%%%%%%%%%%%%%%%%%%%
% Zielsetzung
%%%%%%%%%%%%%%%%%%%%%%%%%%%%%%%%%%%%%%%%%%%%%%%%%%%%%%%%%%%%%%%%%%%%%%%%%%%%%%%
\section{Zielsetzung}
\label{sec:Zielsetzung}

Übergeordnetes Ziel dieser Arbeit ist es, einen neuartigen Ansatz für die Terraingenerierung zu entwickeln, welcher Potenziale von Diffusionsmodellen in diesem Bereich offenbart. Dieser soll intuitiv kontrollierbare, realistische, unendliche Landschaften in der Form von Digital Elevation Models (DEMs) \acused{DEM} generieren können. Zugrunde liegt dabei ein eigens erstelltes latentes Diffusionsmodel. Aus dieser umfassenden Zielstellung lassen sich folgende Kernziele extrahieren:

\begin{enumerate}
    \item \textbf {Framework für Traingsdatenerstellen anhand geographischer Daten:} \\
    Es wird ein Framework entwickelt, welches die Kombination unterschiedlicher geographischer Datensätze zu einem einzigen Testdatensatz so einfach wie möglich gestaltet. Die Quelldaten sollten dabei stets austauschbar sein, um künftiges Experimentieren und Erweitern leicht zu ermöglichen.

    \item \textbf {Entwicklung eines leistungsfähigen Variatonal-Autoencoders:} \\
    Der \ac{VAE} ist eine Kernkomponente in einem latenten Diffusionsmodell. Er übernimmt die Kompression der \ac{DEM}s in einen deutlich geringerdimensionalen latenten Raum, sodass das rechenintensive Diffusionsmodell auf kleineren Daten arbeiten kann. Ebenso ist er verantwortlich für die Abbildung der Ergebnisse zurück in den ursprünglichen Bildraum. Die Qualität der generierten Heightmaps ist also in höchstem Maße von der Umsetzung des \ac{VAE}s abhängig.

    \item \textbf {Erstellung eines mächtigen latenten Diffusionsmodells:} \\
    Basierend auf dem vortrainierten \ac{VAE} soll ein Diffusionsmodell erstellt und trainiert werden. Diese Kombination ergibt ein latentes Diffusionsmodell. Dieses soll anhand von Skizzen und Kontrollsignalen gesteuert werden können. Die Ergebnisse müssen der ursprünglichen Datenverteilung möglichst nahekommen, damit der \ac{VAE} sie zu hochwertigen \ac{DEM}s entschlüsseln kann.
    
    \item \textbf {Intuitive Steuerung der Generierung durch Skizzen:} \\
    Der Generator soll in der Lage sein, auf Basis von einfachen \ac{DEM}-Skizzen realistische und vielfältige Terrains zu generieren. Zentral hierbei ist, dass die Skizze und das Resultat das gleiche Format haben. Durch diese eins-zu-eins Übertragung wird eine intuitive Kontrolle sichergestellt.

    \item \textbf {Nahtlose zellbasierte unendliche Generierung:} \\
    Letztlich soll der Generator in der Lage sein, Landschaftszellen in einem übergordnetem Raster so zu generieren, dass sie nahtlos in einander übergehen. Dies soll auch dann noch einwandfrei funktionieren, wenn benachbarte Zellen starke Unterschiede aufweisen. So wird die Generierung eines beliebig großen und somit theoretisch unendlichen \ac{DEM} erreicht.
     
\end{enumerate}

Durch die Umsetzung dieser Ziele soll ein latentes Diffusionsmodell entstehen, welches intuitiv kontrollierbare, realistische und unendliche Landschaften in Form von \ac{DEM}s generieren kann.

\clearpage

%%%%%%%%%%%%%%%%%%%%%%%%%%%%%%%%%%%%%%%%%%%%%%%%%%%%%%%%%%%%%%%%%%%%%%%%%%%%%%%
% Struktur
%%%%%%%%%%%%%%%%%%%%%%%%%%%%%%%%%%%%%%%%%%%%%%%%%%%%%%%%%%%%%%%%%%%%%%%%%%%%%%%
\section{Struktur}

Diese Arbeit ist in sieben Kapitel gegliedert. Im zweiten Kapitel werden die für das Verständnis der folgenden Inhalte erforderlichen Konzepte und Technologien vorgestellt. \\
Folgend werden in Kapitel drei die entwickelten Ansätze und Methoden detailliert beschrieben. Kapitel vier widmet sich den für das Training des Diffusionsmodells nötigen Datensätzen sowie ihrer Verarbeitung. Hieran anknüpfend werden im fünften Kapitel die wesentlichen Implementierungsdetails und Herausforderungen erläutert. Die aus dem erarbeiten Ansatz resultierenden Ergebnisse werden im sechsten Kapitel präsentiert und anschließend ausführlich diskutiert. \\
Abschließend fasst Kapitel sieben die gewonnen Erkenntnisse zusammen und gibt einen Ausblick auf mögliche zukünftige Entwicklungen und Potenziale des vorgestellten Ansatzes.
