\chapter{Zusammenfassung}

In diesem Kapitel werden die in dieser Arbeit erarbeiteten Ergebnisse und gewonnenen Erkenntnisse in einem Fazit zusammengefasst. Folgend wird diese Arbeit durch einen Ausblick, welcher Perspektiven für weitere Forschung und Experimente auf dem Gebiet der Terraingenierung mit \ac{LDM}s bietet, abgeschlossen.

%%%%%%%%%%%%%%%%%%%%%%%%%%%%%%%%%%%%%%%%%%%%%%%%%%%%%%%%%%%%%%%%%%%%%%%%%%%%%%%
\section{Fazit}

Terraingenerierung mit Diffusionsmodellen ist ein neues und bisher nur wenig erforschtes Gebiet. Diese Arbeit hatte zum Ziel die Eignung dieser Technologie unter Nutzung von auf dem Stand der Technik basierenden Methoden in unterschiedlichen Anwendungsgebieten zu prüfen. \\
Zur Erfüllung dieser Zielstellung wurden zunächst ein \ac{LDM} konzipiert und im zugedessen ein leistungsfähiges VAE-GAN implementiert und trainiert. Basierend hierauf wurde ein \ac{DM} erarbeitet wobei die zwei in der Praxis üblichen Architekturen des \ac{DiT} und U-Nets verglichen wurden. Hieraus wurde erkenntlich, dass ein \ac{DiT} wesentliche Artefakte erzeugt und somit für dieses Anwendungsgebiet in der geprüften Form ungeeignet ist. \\
Auf diesem erarbeiteten \ac{LDM} aufbauend wurden Prozesse für unterschiedliche Anwendungsgebiete in der Terraingenerierung konzipiert. Hierbei stand im Vordergrund die Flexibilität des Einsatzes von \ac{LDM}s zu testen. Konkret wurden hierzu drei Bereiche betrachtet: Unkontrollierte Generierung, Generierung auf der Basis von Skizzen und unendliche Generierung. \\
Das erste dieser Gebiete wies wesentliche Mängel in der Qualität auf. Es wurde vermutet, dass diese auf die Simplifizierungen des Kontrollsignals zurückzuführen sind, welche für eine intuitivere Skizzenkontrolle unternommen wurden. So ist anzunehmen, dass diese Kontrollsignale für eine kohärente globale Struktur der Samples nicht ausreichend aussagekräftig sind. \\
Eben diese Vereinfachungen ermöglichen es jedoch mittels Skizzen, welche direkt auf \ac{DEM}s basieren und die Steuerung der Synthese durch eine \enquote{Kaperng} des Rückwärtsprozesses umsetzen, zu verwenden. Somit werden, im Gegensatz in bisherigen Publikationen, hierbei keine Landschaftssignaturen verwendet, auf welchen das Modell zusätzlich konditioniert werden müsste. Hierdruch ist nicht nur eine intuitivere Kontrolle, da keine Abstraktionsebene vorhanden ist, ermöglicht, sondern auch eine deutlich genauere Kontrolle der Höhenwerte. Zusätlich ist dieser Ansatz über eine dynamische Angabe des Startzeitschrittes sowohl für sehr grobe, als auch sehr detaillierte Skizzen im gleichen Maße geeignet. Desweiteren ist es somit auch möglich Skizzen zu verwenden, welche nicht durch Signaturen abgebildet werden können wie beispielsweise Perlin-Rauschen. \\
Abschließend wurde ein neuartiger Ansatz für die Generierung unendlicher Landschaften erarbeitet, welcher auf der für \ac{DM}s bekannten Technik des Inpaintings aufbaut. Hierbei wird Gitterbasiert anhand bereits generierter benachbarter Zellen ein sich an die Ränder nahtlos und kohärent eingliedernder Landschaftsabschnitt generiert. Ermöglicht wird dies durch die Nutzung einer geeigneten Maskendefinition, für welche in dieser Arbeit eine auf dem Sinus basierende Funktion vorgeschlagen wurde. \\
Die erzielten Ergebnisse in diesen Bereichen demonstrieren eindrucksvoll die flexible Einsatzfähigkeit von \ac{DM}s beziehungsweise \ac{LDM}s für die Terraingenerierung, wenngleich sie auch aufweisen, dass das Potenzial noch bei Weitem nicht ausgeschöpft zu sein scheint.


%%%%%%%%%%%%%%%%%%%%%%%%%%%%%%%%%%%%%%%%%%%%%%%%%%%%%%%%%%%%%%%%%%%%%%%%%%%%%%%
\section{Ausblick}

Durch die Betrachtung der Ergebnisse haben sich einige Bereiche aufgetan, die sich für den weitere Entwicklung der hier gelegten Grundlagen anbieten.  

\subsubsection{Verbesserte Kontrollsignale}

Die wohl gravierendste Limitation die durch das in dieser Arbeit beschriebene \ac{LDM} resultiert ist die minderwertige Qualität von ungesteuerten Synthesen, welche nur durch Kontrollsignale beeinflusst werden. Dies liegt vermutlich an der Simplifizierung der Konditionen im Vergleich zu bisherigen Ansätzen welche nun nicht mehr alleinig für eine kohärente globale Struktur der Terrains ausreichen. \\
Hieraus ergeben sich zwei denkbare Lösungsansätze welche für weitere Untersuchungen geeignet wären. Zum einen eine hierarchische Struktur mehrerer \ac{LDM} welche auf unterschiedlichen Auflösungen der Terrainquelldaten trainiert werden. Dies würde erlauben ein Modell auf groben Daten zu Trainieren, welches somit ohne Kenntniss von kleinen Details die Verteilung der großflächigen Eigenschaften einer Landschaft erlernt. Zusätzlich könnte dies zur Generierung von unterschiedlichen Levels of Detail dienen, um einer Generierung zur Laufzeit näher zu kommen.\\
Zum anderen könnten mit unterschiedlichen strengeren Kontrollsignalen experimentiert werden, wobei hierbei zu beachten ist, dass diese die skizzenbasierte Generierung nicht negativ beeinträchtigen.  

\subsubsection{Alternative Samplingalgorithmen}

Aktuell sieht das implementierte Modell vor, dass, dem normalen DDPM-Rückwärtsprozess entsprechend, für das Sampling alle $T$ Zeitschritte durchlaufen werden müssen. Dies ist Zeit- und Rechenintensiv. Aus diesem Grund existieren bereits viele alternative Samplingalgorithmen welche darauf abzielen mit weniger Zeitschritten gute Ergebnisse zu erzeugen, wie DDIM oder DPM++. Die Eignung dieser Algorithmen zu überprüfen ist eine vielversprechende Option um die nötige Rechenzeit drastisch zu reduzieren.

\subsubsection{Flow Matching}

In jüngster Forschung hat sich eine der Diffusion sehr verwandte Technik der generativen \ac{KI}, das sogenannte Flow Matching\footnote{
    Lipman et al.: Flow Matching for Generative Modelling
    \cite{lipman2023flowmatchinggenerativemodeling}
} entwickelt. Dieses verspricht schnellere Ergebnisse und höhere Samplequalität. Aufgrund der unmittelbaren Nähe zur Diffusion ist davon auszugehen, dass sich diese Technik ebenfalls für die Terraingenerierung eignet. Eine genauere Untersuchung hiervon könnte die Leistungsfähigkeit generativer \ac{KI} in diesem Bereich noch weiter erhöhen.  